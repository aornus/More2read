\section{七月十六日}
\dailyquote
{He most lives who thinks most, feels the noblest and acts the best. 

思想练达,情操高尚,行为优雅的人是生活地最好的人。}
{Samuel Bailey(塞缪尔·贝利 )}
{monkey}
\poem{Solitude(孤独)}
{Ella Wheeler Wilcox(埃拉·惠勒·威尔科克斯)}
{Laugh, and the world laughs with you;
	
	Weep, and you weep alone.
	
	For the sad old earth must borrow it's mirth,
	
	But has trouble enough of it's own.\\
	
	
	Sing, and the hills will answer;
	
	Sigh, it is lost on the air.
	
	The echoes bound to a joyful sound,
	
	But shrink from voicing care.\\
	
	
	Rejoice, and men will seek you;
	
	Grieve, and they turn and go.
	
	They want full measure of all your pleasure,
	
	But they do not need your woe.\\
	
	
	Be glad, and your friends are many;
	
	Be sad, and you lose them all.
	
	There are none to decline your nectared wine,
	
	But alone you must drink life's gall.\\
	
	
	Feast, and your halls are crowded;
	
	Fast, and the world goes by.
	
	Succeed and give, and it helps you live,
	
	But no man can help you die.\\
	
	
	There is room in the halls of pleasure
	
	For a long and lordly train,
	
	But one by one we must all file on
	
	Through the narrow aisles of pain.}
{你欢笑,这世界陪你一起欢笑;
	
	你哭泣,却只能独自黯然神伤。
	
	只因古老而忧伤的大地必须注入欢乐,
	
	它的烦恼已经足够。\\
	
	
	你歌唱,山谷将与你合音共曲;
	
	你叹息,空气将其无声埋没。
	
	快乐之声总能引起回声阵阵,
	
	忧虑之音却令其销声匿迹。\\
	
	
	你欢喜,人们会与你相随;
	
	你悲伤,人们则会转身离去。
	
	人人都愿分享你全部的快乐,
	
	但无人需要你的哀伤苦楚。\\
	
	
	快乐起来吧,你会拥有众多朋友。
	
	若悲伤感怀,你会与他们失之交臂。
	
	没有人会拒绝与你共酌美酒甘露,
	
	然而生活的苦汁你必须独自品尝。\\
	
	
	饕餮盛宴,你的厅堂人潮涌动,
	
	时光飞逝,这世界匆匆而过,
	
	成功、给予,是你生活的动力,
	
	然而没有人能够替你离世而去。\\
	
	
	因为欢愉的殿堂里,
	
	容得下一节长长的豪华火车,
	
	但我们必须一个接一个地排队,
	
	穿越岁月磨难的狭长隧道而进入。}
\words
{	猴子的故事(The Story of the Monkey)}
{梁漱溟(Shuzhen liang)}
{人类顶大的长处是智慧,但什么是智慧呢?智慧有一个要点,就是要冷静。譬如:正在计算数目,思索道理的时候,如果心里气恼,或喜乐、或悲伤,必致错误或简直不能进行。这是大家都明白的事。却是一般人对于解决社会问题,偏不明此理。他们总是为感情所蔽,而不能静心体察事理,从事理中寻出解决的办法。像是军阀问题,麻木不仁者不去关心;去想这问题的人,便不胜其憎恶排斥之情,不复能分析研究其所从来。那么,想出的办法,就不外是打倒军阀之类了。又如要求国家统一的人,不能分析研究中国陷于不统一的由来,总是急切地要求统一,那么就以武力来统一了。然而打倒军阀者,试问可曾打倒没有呢?以武力求统一者,试问统一了没有呢?
	
	Man's greatest strength is wisdom, but what is wisdom? The key to wisdom is the ability to remain composed. For example, when you are calculating numbers or pondering over problem ,if you are emotionally disturbed by anger, joy or grief, you are bound to make mistakes, or you may not even be able to proceed. This is something everyone understand this. They are always taken over by their emotions and thus are unable to calm down to observe how things work and find solutions to their problems.
	
	我想说一个猴子的故事给大家听。在汤姆孙科学大纲上叙说一个科学家研究动物心理,养着几个猩猩猴子作实验。以一个高的玻璃瓶,拔去木塞,放两粒花生米进去。花生米自然落到瓶底,从玻璃外面可以看见,递给猴子。猴子接过,乱摇许久,偶然摇出花生米来,才得取食。此科学家又放花生米如前,而指教他只须将瓶子一倒转,花生米立刻出来。但是猴子总不理会他的指教,每次总是乱摇,很费力气而不能必得。此时要研究猴子何以不能领受人的指教呢?没有旁的,只为他两眼看见花生米,一心急切求食,就再无余暇来理解与学习了。要学习,必须他两眼不去看花生米,而移其视线来看人的手势与瓶子的倒转,才行。要转移视线,必须他平下心去,不为食欲冲动所蔽,才行。然而他竟不会也。猴子的智慧的贫乏,就在此等处。
	
	I would like to share with you a story of the monkey. In Thompson's The Outline of Science there is a story about a scientist who kept several chimpanzees and monkeys in order to study animal psychology. He took a glass bottle, removed its cork and put two peanuts inside the bottle. Needless to say, the peanuts dropped to the bottom of the bottle and were easily seen from the outside. He then passed the bottle to a monkey, who shook it frantically for a long while and was only able to get the peanuts when they accidentally fell out. The scientist then put some into the bottle again as he had done before and showed the monkey that it only needed to turn the bottle upside down for the peanuts to drop out. But the monkey always ignored his instructions. Each time it just shook the bottle frantically, with great effort but without necessarily achieving the desired result. Now the question is why the monkey was unable to understand the instruction of the scientist. Simply because all its attention was focused on the peanuts.
	
	人们不感觉问题,是麻痹;然为问题所刺激,辄耐不住,亦不行。要将问题放在意识深处,而游心于远,从容以察事理。天下事必能了解他,才能控制他。情急之人何以异于猴子耶?
	
	
	As it was single-mindedly concentrating on reaching the food, it had no time for understanding and learning. To learn, it must take its eyes off the peanuts and shift its attention to the hand movement of the man and the way the bottle was turned upside down. To shift its attention, it had to claim down and not be taken over by the impulse of its appetite. Yet the monkey was not able to understand this. It is instances like this that reveal the monkey's lack of wisdom.
	Some people don't sense problems because they are insensitive; yet it is also wrong to be excited by a problem and lose self-control. One cannot control things unless one has understood them first. What difference is there between an impatient human being and monkey?
	
	还要注意:人的心思,每易从其要求之所指而思索办法;观察事理,亦顺着这一条线而观察。于是事理也,办法也,随着主观都有了。其实只是自欺,只是一种自圆其说。智慧的优长或贫乏,待看他真冷静与否。
	
	There is one more thing to note: people tend to seek solutions on the basis of what they need and make observations in the same manner, believing that understanding and solutions will follow their subjective thinking. In fact this is merely self-deception or self-indulgence. The possession or lack of wisdom depends on whether the person involved is really able to calm down and remain detached.}