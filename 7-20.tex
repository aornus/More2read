% \section{title}
% \dailyquote{中英}{作者}{封面}
% \poem{作品名}{作者}{英文}{中文}
% \words{作品名}{作者}{内容}
\section{七月二十日}
 \dailyquote{Silence in the face of evil is itself evil. 
 	
 	面对邪恶时的沉默不言,本身就是邪恶。

}{Dietrich Bonhoeffer(迪特里希·朋霍费尔 )}{Walden}
%\footnote{德国人,(1906年2月4日—1945年4月9日),德国信义宗牧师,认信教会的创始人之一,也是一名神学家。出生在德国布雷斯劳(今波兰弗罗茨瓦夫)。曾经参加在德国反对纳粹主义的抵抗运动,并计划刺杀希特勒。在1943年3月被拘捕,最后在二次大战结束前被绞死,而希特勒于不久后自杀身亡。}
\poem{蝉(To the Cicada)}{(唐)虞世南}{	
	Though rising high, you drink but dew,
	
	Yet your voice flows from sparse plane trees.
	
	Far and wide there's none but hears you,
	
	You need no wings of autumn breeze.}{
	垂緌饮清露,
	
	流响出疏桐。
	
	居高声自远,
	
	非是藉秋风。	
}
\words{Walden(瓦尔登湖节选)}{Henry David Thoreau}{
	\footnote{	本文节选自瓦尔登湖中「Where I Lived, and What I Lived for」一章,网上找了一圈也没有找到播客里的那个译本,找来的这篇文风比较粗劣,但凑合着也能看\faMehBlank[regular]。}
I have frequently seen a poet withdraw, having enjoyed the most valuable part of a farm, while the crusty farmer supposed that he had got a few wild apples only. Why, the owner does not know it for many years when a poet has put his farm in rhyme, the most admirable kind of invisible fence, has fairly impounded it, milked it, skimmed it, and got all the cream, and left the farmer only the skimmed milk.

我时常看到一个诗人,在欣赏了一片田园风景中的最珍贵部分之后,就扬长而去,那些固执的农夫还以为他拿走的仅只是几枚野苹果。诗人却把他的田园押上了韵脚,而且多少年之后,农夫还不知道这回事,这么一道最可羡慕的、肉眼不能见的篱笆已经把它圈了起来,还挤出了它的牛乳,去掉了奶油,把所有的奶油都拿走了,他只把去掉了奶油的奶水留给了农夫。


The real attractions of the Hollowell farm, to me, were: its complete retirement, being, about two miles from the village, half a mile from the nearest neighbor, and separated from the highway by a broad field; its bounding on the river, which the owner said protected it by its fogs from frosts in the spring, though that was nothing to me; the gray color and ruinous state of the house and barn, and the dilapidated fences, which put such an interval between me and the last occupant; the hollow and lichen-covered apple trees, nawed by rabbits, showing what kind of neighbors I should have; but above all, the recollection I had of it from my earliest voyages up the river, when the house was concealed behind a dense grove of red maples, through which I heard the house-dog bark. I was in haste to buy it, before the proprietor finished getting out some rocks, cutting down the hollow apple trees, and grubbing up some young birches which had sprung up in the pasture, or, in short, had made any more of his improvements. To enjoy these advantages I was ready to carry it on; like Atlas, to take the world on my shoulders — I never heard what compensation he received for that — and do all those things which had no other motive or excuse but that I might pay for it and be unmolested in my possession of it; for I knew all the while that it would yield the most abundant crop of the kind I wanted, if I could only afford to let it alone. But it turned out as I have said.

霍乐威尔田园的真正迷人之处,在我看是:它的遁隐之深,离开村子有两英里,离开最近的邻居有半英里,并且有一大片地把它和公路隔开了;它傍着河流,据它的主人说,由于这条河,而升起了雾,春天里就不会再下霜了,这却不在我心坎上;而且,它的田舍和棚屋带有灰暗而残败的神色,加上零落的篱笆,好似在我和先前的居民之间,隔开了多少岁月;还有那苹果树,树身已空,苔藓满布,兔子咬过,可见得我将会有什么样的一些邻舍了;但最主要的还是那一度回忆,我早年就曾经溯河而上,那时节,这些屋宇藏在密密的红色枫叶丛中,还记得我曾听到过一头家犬的吠声。我急于将它购买下来,等不及那产业主搬走那些岩石,砍伐掉那些树身已空的苹果树,铲除那些牧场中新近跃起的赤杨幼树,一句话,等不及它的任何收拾了。为了享受前述的那些优点,我决定干一下了;像那阿特拉斯[1]一样,把世界放在我肩膀上好啦,——我从没听到过他得了哪样报酬,——我愿意做一切事:简直没有别的动机或任何推托之辞,只等付清了款子,便占有这个田园,再不受他人侵犯就行了;因为我知道我只要让这片田园自生自展,它将要生展出我所企求的最丰美的收获。但后来的结果已见上述。


All that I could say, then, with respect to farming on a large scale — I have always cultivated a garden — was, that I had had my seeds ready. Many think that seeds improve with age. I have no doubt that time discriminates between the good and the bad; and when at last I shall plant, I shall be less likely to be disappointed. But I would say to my fellows, once for all, As long as possible live free and uncommitted. It makes but little difference whether you are committed to a farm or the county jail.

所以,我所说的关于大规模的农事(至今我一直在培育着一座园林),仅仅是我已经预备好了种子。许多人认为年代越久的种子越好。我不怀疑时间是能分别好和坏的,但到最后我真正播种了,我想我大约是不至于会失望的。可是我要告诉我的伙伴们,只说这一次,以后永远不再说了:你们要尽可能长久地生活得自由,生活得并不执著才好。执迷于一座田园,和关在县政府的监狱中,简直没有分别。


Old Cato, whose "De Re Rusticâ" is my "Cultivator," says — and the only translation I have seen makes sheer nonsense of the passage — "When you think of getting a farm turn it thus in your mind, not to buy greedily; nor spare your pains to look at it, and do not think it enough to go round it once. The oftener you go there the more it will please you, if it is good." I think I shall not buy greedily, but go round and round it as long as I live, and be buried in it first, that it may please me the more at last.

老卡托——他的《乡村篇》是我的“启蒙者”,曾经说过——可惜我见到的那本唯一的译本把这一段话译得一塌糊涂,——“当你想要买下一个田园的时候,你宁可在脑中多多地想着它,可决不要贪得无厌地买下它,更不要嫌麻烦而再不去看望它,也别以为绕着它兜了一个圈子就够了。如果这是一个好田园,你去的次数越多你就越喜欢它。”我想我是不会贪得无厌地购买它的,我活多久,就去兜多久的圈子,死了之后,首先要葬在那里。这样才能使我终于更加喜欢它。


The present was my next experiment of this kind, which I purpose to describe more at length, for convenience putting the experience of two years into one. As I have said, I do not propose to write an ode to dejection, but to brag as lustily as chanticleer in the morning, standing on his roost, if only to wake my neighbors up.

目前要写的,是我的这一类实验中其次的一个,我打算更详细地描写描写;而为了便利起见,且把这两年的经验归并为一年。我已经说过,我不预备写一首沮丧的颂歌,可是我要像黎明时站在栖木上的金鸡一样,放声啼叫,即使我这样做只不过是为了唤醒我的邻人罢了。


When first I took up my abode in the woods, that is, began to spend my nights as well as days there, which, by accident, was on Independence Day, or the Fourth of July, 1845, my house was not finished for winter, but was merely a defence against the rain, without plastering or chimney, the walls being of rough, weather-stained boards, with wide chinks, which made it cool at night. The upright white hewn studs and freshly planed door and window casings gave it a clean and airy look, especially in the morning, when its timbers were saturated with dew, so that I fancied that by noon some sweet gum would exude from them. To my imagination it retained throughout the day more or less of this auroral character, reminding me of a certain house on a mountain which I had visited a year before. This was an airy and unplastered cabin, fit to entertain a travelling god, and where a goddess might trail her garments. The winds which passed over my dwelling were such as sweep over the ridges of mountains, bearing the broken strains, or celestial parts only, of terrestrial music. The morning wind forever blows, the poem of creation is uninterrupted; but few are the ears that hear it. Olympus is but the outside of the earth everywhere.

我第一天住在森林里,就是说,白天在那里,而且也在那里过夜的那一天,凑巧得很,是一八四五年七月四日,独立日,我的房子没有盖好,过冬还不行,只能勉强避避风雨,没有灰泥墁,没有烟囱,墙壁用的是饱经风雨的粗木板,缝隙很大,所以到晚上很是凉爽。笔直的、砍伐得来的、白色的间柱,新近才刨得平坦的门户和窗框,使屋子具有清洁和通风的景象,特别在早晨,木料里饱和着露水的时候,总使我幻想到午间大约会有一些甜蜜的树胶从中渗出。这房间在我的想象中,一整天里还将多少保持这个早晨的情调,这使我想起了上一年我曾游览过的一个山顶上的一所房屋。这是一所空气好的、不涂灰泥的房屋,适宜于旅行的神仙在途中居住,那里还适宜于仙女走动,曳裙而过。吹过我的屋脊的风,正如那扫荡山脊而过的风,唱出断断续续的调子来,也许是天上人间的音乐片段。晨风永远在吹,创世纪的诗篇至今还没有中断;可惜听得到它的耳朵太少了。灵山只在大地的外部,处处都是。
}