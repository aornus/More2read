 \section{七月二十二日}
 \dailyquote{Time cures sorrows and squabbles because we all change, and are no longer the same persons. Neither the offender nor the offended is the same. 

时间可以消除忧虑和争吵,因为 我们大家都在\textbf{改变},不再和从前一样,触犯者与被触犯者都不是曾经的那个人了。
 }{Blaise Pascal(帕斯卡)\footnote{帕斯卡出生于1623年,1662年去世。法国数学家、物理学家、哲学家、散文家。
%16岁时他发现了著名的帕斯卡六边形定理。17岁时写成圆锥曲线论,这些工作是自希腊阿波罗尼奥斯以来圆锥曲线论的最大进步。1642年,他设计并制作了一台能自动进位的加减法计算装置,被称为是世界上第一台数字计算机,为以后的计算机设计提供了基本原理。1654年,他开始研究几个方面的数学问题,在无穷小分析上,深入探讨了不可分原理,得出求不同曲线所围面积和重心的一般方法,并以积分学的原理解决了摆线问题,于1658年,完成「论摆线」,他的论文手稿对莱布尼茨建立微积分学有很大启发。自1655年隐居修道院后,他写下了「思想录」等经典著作
}}{habit}
 \poem{Remember Me When I am Gone Away(若我离去,请把我记起)}
 {\footnote{取自「小妖精集市」}  Christina Rossetti (克里丝蒂娜·罗塞蒂)}
 {Remember me when I am gone away,
 	
 	Gone far away into the silent land;
 	
 	When you can no more hold me by the hand,
 	
 	Nor I half turn to go yet turning stay.\\
 	
 	Remember me when no more day by day
 	
 	You tell me of our future that you plann'd:
 	
 	Only remember me; you understand
 	
 	It will be late to counsel then or pray.\\
 	
 	
 	
 	Yet if you should forget me for a while
 	
 	And afterwards remember, do not grieve:
 	
 	For if the darkness and corruption leave
 	
 	A vestige of the thoughts that once I had,
 	
 	Better by far you should forget and \textbf{smile}
 	
 	Than that you should remember and be sad.}
 {若我离去,请把我记起,
 	
  当我去往遥远的静谧之地;
  
  那时  你再不能把我的手握起,
  
  我也不能转身  要走  却又迟疑。\\
  


请把我记起,当你再不能日复一日

对我描绘我们未来的生活:

只望你记得我,因为你懂得

那时交谈或祈祷都已为时太迟。\\


但若可以,是否暂时把我忘记,

之后再把我想起,请不必忧伤:

因为,如果黑暗和腐朽的地方

能让我曾经的思绪 有些许痕迹,

但求你,笑着把我忘却

也好过悲伤地将我记起。

}
\music{ Moonlight Shadow(月影)}
%跟Enya(恩雅)的歌很像
{
翻		\includemedia[
transparent,
activate=onclick,
passcontext,
flashvars={
	source=MoonlightShadow.mp3
	&loop=true	}
]{唱}{MoonlightShadow.mp3}:Dana Winner(丹娜·云妮)\\
作词作曲:Mike Oldfield(麦克·欧菲尔德)
}
{\footnote{最近听了这首歌,感觉与上面的那首诗很相称,空灵与忧伤的感觉}
The last that ever she saw him

Carried away by a moonlight shadow

He passed on worried and warning

Carried away by a moonlight shadow.

Lost in a river last saturday night

Far away on the other side.

He was caught in the middle of a desperate fight

And she couldn't find how to push through\\


The trees that whisper in the evening

Carried away by a moonlight shadow

Sing a song of sorrow and grieving

Carried away by a moonlight shadow

All she saw was a silhouette of a gun

Far away on the other side.

He was shot six times by a man on the run

And she couldn't find how to push through\\

I stay

I pray

I see you in heaven one day\\

Four am in the morning

Carried away by a moonlight shadow

I watched your vision forming

Carried away by a moonlight shadow

Star was light in a silvery night

Far away on the other side

Will you come to talk to me this night

But she couldn't find how to push through\\


I stay

I pray

I see you in heaven one day\\

Far away on the other side.\\


Caught in the middle of a hundred and five

The night was heavy but the air was alive

But she couldn't find how to push through\\

Carried away by a moonlight shadow

Carried away by a moonlight shadow

Far away on the other side. \\

}
{
那是她最后一次见到他

因月之阴影而悄然离去

他消逝于忧虑和警示中

因月之阴影而悄然离去

他沉溺于上周末的河中

在遥远的那边消遁无形

他就这样死于这场决斗中

而她不知道之后该如何度过\\


树林在黄昏时分的低语

因月之阴影而悄然远去

唱一首悲伤的挽歌吧

因月之阴影而悄然远去

她看到的是一只枪的侧影

在遥远的那边慢慢举起

一个逃跑的男人向他开了六枪

而她不知道之后该如何度过\\



我止步

我祈祷

我看到你在天堂渐渐远去\\


凌晨四点钟的时间

因月之阴影而悄然逝去

我仿佛看见你的幻像

因月之阴影而悄然逝去

银色的夜里星光熠熠

在遥远的那边静静闪烁

今天晚上你还会回来找我吗?

而她不知道之后该如何度过\\


我止步

我祈祷

我看到你在天堂渐渐远去\\


一直到遥远的那边\\


悲伤的人群伴随着他

夜色凝重而空气仍在流动

而她不知道之后该如何度过\\


因月之阴影而远去

因月之阴影而远去

一直到遥远的那边 
\begin{flushright}
	——萧饮寒译
\end{flushright}
}
 \words{养成好习惯( Cultivating Good Habits )}{ 梁实秋(Shiqiu Liang)}{
  人的天性大致是差不多的,但是在习惯方面却各有不同,习惯是慢慢养成的,在幼小的时候最容易养成,一旦养成之后,要想改变过来却还不很容易。
 
 Men are about the same in human nature, but differ in habit. Habit is formed little by little, and most easily in one's childhood. Once it is formed, it is difficult to break.
 
 例如说:清晨早起是一个好习惯,这也要从小时候养成,很多人从小就贪睡懒觉,\textbf{一遇假日便要睡到日上三竿还高卧不起,}平时也是不肯早起,往往蓬首垢面的就往学校跑,结果还是迟到,这样的人长大了之后也常是不知振作,多半不能有什么成就。祖逖闻鸡起舞,那才是志士奋励的榜样。
 
 For example, the good habit of early rising also starts from one's early life. Many people, however, have been in the habit of sleeping late ever since they were kids. They won't get up till late morning on holidays and even oversleep on work days. Children are often late for school though they make a rush even without washing up. Such children, when they grow up, will often lack drive and most probably get nowhere. The story of Zu Ti1 rising at cockcrow to practise swordplay should be a good example for all men of resolve to learn from.
 
 我们中国人最重礼,因为礼是行为的轨范。礼要从家庭里做起。姑举一例:为子弟者“出必告,反必面”,这一点点对长辈的起码的礼,我们是否已经每日做到了呢?我看见有些个孩子们早晨起来对父母视若无睹,晚上回到家来如入无人之境,遇到长辈常常横眉冷目,不屑搭讪。这样的跋扈乖戾之气如果不早早的纠正过来,将来长大到社会服务,必将处处引起摩擦不受欢迎。我们不仅对长辈要恭敬有礼,对任何人都应维持相当的礼貌。
 
 We Chinese set great store by propriety because it is the accepted rules of social behavior. Propriety begins from the family. For example, children should keep their parents informed of their whereabouts. That is the ABC of good manners on the part of children. Yet some children just ignore their parents when they get up in the morning or come back from school. They often pull a long face and refuse to converse when they meet their elders. If they continue to be so cocky and willful without correcting themselves as soon as possible, they will never get along well with other people some days as members of society. We should be polite not only to our elders, but also to all people.
 
% 大声讲话,扰及他人的宁静,是一种不好的习惯。我们试自检讨一番,在别人读书工作的时候是否有过喧哗的行为?我们要随时随地为别人着想,维持公共的秩序,顾虑他人的利益,不可放纵自己,在公共场所人多的地方,要知道依次排队,不可争先恐后地去乱挤。
% 
% We Chinese set great store by propriety because it is the accepted rules of social behavior. Propriety begins from the family. For example, children should keep their parents informed of their whereabouts. That is the ABC of good manners on the part of children. Yet some children just ignore their parents when they get up in the morning or come back from school. They often pull a long face and refuse to converse when they meet their elders. If they continue to be so cocky and willful without correcting themselves as soon as possible, they will never get along well with other people some days as members of society. We should be polite not only to our elders, but also to all people.
 
 时间即是生命。我们的生命是一分一秒的在消耗着,我们平常不大觉得,细想起来实在值得警惕。我们每天有许多的零碎时间于不知不觉中浪费掉了,我们若能养成一种利用闲暇的习惯,一遇空闲,无论其为多么短暂,都利用之做一点有益身心之事,则积少成多终必有成。
 
 Time is life. Our life is ticking away unnoticed minute by minute and second by second. It is certainly alarming when we come to think of it. Every day we are unconsciously wasting many odd moments. We should acquire the habit of utilizing leisure time, and snatch every odd moment to do whatever is beneficial to our body and mind. That will enable us to achieve good results little by little. 
 
 常听人讲过“消遣”二字,最是要不得,好像是时间太多无法打发的样子,其实人生短促极了,哪里会有多余的时间待人“消遣”?陆放翁有句云:「待饭未来还读书。」我知道有人就经常利用这“待饭未来”的时间读了不少的大书。古人所谓「三上之功」,枕上、马上、厕上,虽不足为训,其用意是在劝人不要浪费光阴。
 
 People often talk most improperly about "seeking relaxation" as if they had more than enough time for them to while away. Life is, in fact, extremely short. How can you find so much surplus time for you to fool away? Lu Fangweng says in one of his poems, "\textit{Spend even the pre-meal odd moment in reading.}" As far as I know, many people did snatch the odd moment before a meal to do a lot of reading. Our ancients recommended "\textit{three on's}", that is, doing reading even while you are on a pillow, on a horse or on a nightstool. All that, though impracticable, serves the purpose of advising people not to waste time.
 
 吃苦耐劳是我们这个民族的标志。古圣先贤总是教训我们要能过得俭朴的生活,一个有志的人之能耐得清寒。
% \footnote{此处与播客上有所出入}
 恶衣恶食,不足为耻,丰衣足食,不足为荣,这在个人之修养上是应有的认识。
 
 Ours is a nation known for industry and self-denial. Frugality has always been the teaching of our ancient sages and wise men. A man of strong will should be able to endure Spartan living conditions. It should not be regarded as a disgrace to live a simple life. Nor should it be regarded as a glory to live a luxurious life. That should be the correct understanding one needs for self-cultivation. 
 
 罗马帝国盛时的一位皇帝,Marcus Aurelius,他从小就摒绝一切享受,从来不参观那当时风靡全国的赛车比武之类的娱乐,终其身成为一位严肃的苦修派的哲学家,而且也建立了不朽的事功。这是很值得钦佩的,我们中国是一个穷的国家,所以我们更应该体念艰难,弃绝一切奢侈,尤其是从外国来的奢侈。从小就养成俭朴的习惯,更要知道物力维艰,竹头木屑,皆宜爱惜。
 
 Marcus Aurelius, emperor of the Roman Empire in its heyday, refused to enjoy all comforts of life from childhood and always keep kept away from amusements like the chariot race then in vogue and other fighting-skill competitions. He remained a life-long staunch Stoic philosopher and meanwhile distinguished himself by numerous exploits. Ours is a poor country, so it is even more necessary for us to see the tough conditions facing us and renounce all luxuries, especially those coming from abroad. We should build up the habit of leading a thrifty life. We should bear in mind that all material resources are hard to come by and should be treasured, even including their odds and ends.
 
 以上数端不过是偶然拈来,好的习惯千头万绪,“勿以善小而不为”。习惯养成之后,便毫无勉强,临事心平气和,顺理成章。充满良好习惯的生活,才是合于“自然”的生活。
 
 The above points have been picked by me at random. Good habits are too numerous to be dealt with one by one, but none, however, are too small to keep. Habit, once formed, will become your natural and spontaneous behavior. A life full of good habits will be a life conforming with the law of nature.}