 \section{七月二十五日}
 \dailyquote{ Don't forget until too late that the business of life is not business but living. 
 	
 	要尽早地明白,人生的事务不是在于工作,而是在于生活。}{Bertie Charles Forbes(福布斯)}{read}
 \poem{清平乐·闲居书付儿辈(To My Sons )}{ 	
 	陈继儒\footnote{陈继儒(1558-1639),字仲醇,号眉公、麋公,松江府华亭(今上海市松江区)人。明朝文学家、画家。}(Jiru Chen)}
 {Life is complete
 	
 	With children at your feet;
 	
 	Just a handful of hay hides your cot.
 	
 	If land is sterile,
 	
 	A young calf will surely help a lot.\\
 	
 	
 	Teach thy sons to read, too, in spare hours,
 	
 	Not for fame nor for Mandarin collars.
 	
 	Brew  wine, plant bamboos, water flowers,
 	
 	Thus a house for generations of scholars.
 }
 { 
 	有儿事足,
 	
 	一把茅遮屋。
 	
 	若使薄田耕不熟,
 	
 	添个新生黄犊。\\
 	
 	闲来也教儿孙,
 	
 	读书不为功名。
 	
 	种竹,浇花,酿酒;
 	
 	世家闭户先生。
 	\\
 	\\
 	\\
 	
 }
 \words{Detached Thoughts on Books and Reading(读书漫谈)节选}{Charles Lamb\footnote{查尔斯·兰姆(1775—1834),英国著名散文家和评论家。}}{
\begin{quotation}
	 	To mind the inside of a book is to entertain one's self with the forced product of another man's brain. Now I think a man of quality and breeding may be much amused with the natural sprouts of his own.
 
\begin{flushright}
	 —Lord Foppington in \textit{the Relapse}
\end{flushright}
\end{quotation}
 
\begin{quotation}
	 把心思用在读书上,不过是想从别人绞尽脑汁、苦思冥
 想的结果中找点乐趣。其实,我想,一个有本领、有教养的
 人,灵机一动,自有奇思妙想联翩而来,这也就尽够他自己
 受用的了。
\begin{flushright}
	  ——《旧病复发》中福平顿爵士的台词
\end{flushright}
\end{quotation}

 An ingenious acquaintance of my own was so much struck with this bright sally of his lordship, that he has left off reading altogether, to the great improvement of his originality. At the hazard of losing some credit on this head, I must confess that I dedicate no inconsiderable portion of my time to other people's thoughts. I dream away my life in others' speculations. I love to lose myself in other men's minds. When I am not walking, I am reading; I cannot sit and think. Books think for me.
 
 我认识的一位生性伶俐的朋友,听了爵爷这段出色的俏皮话,在惊佩之余,完全放弃了读书;从此他遇事独出心裁,比往日大有长进。我呢,冒着在这方面丢面子的危险,却只好老实承认:我把相当大一部分时间用来读书了。我的生活,可以说是在与别人思想的神交中度过的。我情愿让自己淹没在别人的思想之中,除了走路,我便读书,我不会坐在那里空想——自有书本替我去想。
 
 
 I have no repugnance. Shaftesbury is not too genteel for me, nor Jonathan Wild too low. I can read anything which I call a book. There are things in that shape which I cannot allow for such.
 
 在读书方面,我百无禁忌。高雅如夏夫茨伯利,低俗如《魏尔德传》,我都一视同仁。凡是我可以称之为“书”的,我都读。但有些东西,虽具有书的外表,我去不把它们当作书看。
 

 In this catalogue of books which are no books—biblia a-biblia—I reckon Court Calendars, Directories, Pocket Books, Draught Boards, bound and lettered on the back, Scientific Treatises, Almanacks, Statutes at Large; the works of Hume, Gibbon, Robertson, Beattie, Soame Jenyns, and, generally, all those volumes which “no gentleman's library should be without”; the histories of Flavius Josephus (that learned Jew), and Paley's Moral Philosophy. With these exceptions, I can read almost anything. I bless my stars for a taste so catholic, so unexcluding.
 
  在bibla -biblia(非书之书)这一类别里,我列入了《宫廷事例年表》、《礼拜规则》、袖珍笔记本、订成书本模样而背面印字的棋盘、科学论文、日历、《法令大全》、休漠、吉本、洛伯森、毕谛、索姆?钱宁斯等人的著作,以及属于所谓“绅士必备藏书”的那些大部头;还有弗来维?约瑟夫斯(那位有学问的犹太人)的历史者作和巴莱的《道德哲学》。把这些东西除外,我差不多什么书都可以读。我庆幸自己命交好运,得以具有如此广泛而无所不包的兴趣。
 
 I confess that it moves my spleen to see these things in books' clothing perched upon shelves, like false saints, usurpers of true shrines, intruders into the sanctuary, thrusting out the legitimate occupants. To reach down a well-bound semblance of a volume, and hope it is some kind-hearted playbook, then, opening what “seem its leaves”, to come bolt upon a withering Population Essay. To expect a Steele, or a Farquhar, and find—Adam Smith. To view a well-arranged assortment of blockheaded Encyclopaedias (Anglicanas or Metropolitanas) set out in an array of Russia, or Morocco, when a tithe of that good leather would comfortably re-clothe my shivering folios; would renovate Paracelsus himself, and enable old Raymund Lully to look like himself again in the world. I never see these impostors, but I long to strip them, to warm my ragged veterans in their spoils.
 
 老实说,每当我看到那些披着书籍外衣的东西高踞在书架之上,我就禁不住怒火中烧,因为这些假圣人篡夺了神龛,侵占了圣堂,却把合法的主人赶得无处存身。从书架上拿下来装订考究、书本模样的一大本,心想这准是一本叫人开心的“大戏考”,可是掀开它那“仿佛书页似的玩意儿”一瞧,却是叫人扫兴的《人口论》。想看看斯梯尔或是法奈尔,找到的都是亚当?史密斯。有时候,我看见那些呆头呆脑的百科全书(有的叫“大英”,有的叫“京都”),分门别类,排列齐整,一律用俄罗斯皮或摩洛哥皮装订,然而,相比之下,我那一批对开本的老书却是临风瑟缩,衣不蔽体——我只要能有那些皮子的十分之一,就能把我那些书气气派派地打扮起来,让派拉塞尔萨斯焕然一新,让雷蒙?拉莱能够在世人眼中恢复本来面目。每当我瞅见那些衣冠楚楚的欺世盗名之徒,我就恨不得把它们身上那些非分的装裹统统扒下来,穿到我那些衣衫槛褛的旧书身上,让它们也好避避寒气。
 

 To be strong-backed and neat-bound is the desideratum of a volume. Magnificence comes after. This, when it can be afforded, is not to be lavished upon all kinds of books indiscriminately. I would not dress a set of Magazines, for instance, in full suit. The dishabille, or half-binding (with Russia backs ever) is our costume. A Shakespeare, or a Milton (unless the first editions), it were mere foppery to trick out in gay apparel. The possession of them confers no distinction.
 
  对于一本书来说,结结实实、齐齐整整地装订起来,是必不可少的事情,豪华与否倒在其次。而且,装订之类即使可以不计工本,也不必对各类书籍不加区别,统统加以精装。譬如说,我就不赞成对杂志合订本实行全精装——简装或半精装(用俄罗斯皮),也就足矣。而把一部莎士比亚或是一部弥尔顿(除非是第一版)打扮得花花绿绿,则是一种纨绔子弟习气。

 The exterior of them (the things themselves being so common), strange to say, raises no sweet emotions, no tickling sense of property in the owner. Thomson's Seasons, again, looks best (I maintain it) a little torn, and dog's-eared. How beautiful to a genuine lover of reading are the sullied leaves, and worn out appearance, nay, the very odour (beyond Russia), if we would not forget kind feelings in fastidiousness, of an old “Circulating Library”Tom Jones, or Vicar of Wakefield! How they speak of the thousand thumbs, that have turned over their pages with delight!—of the lone sempstress, whom they may have cheered (milliner, or harder-working mantua-maker) after her long day's needle-toil, running far into midnight, when she has snatched an hour, ill spared from sleep, to steep her cares, as in some Lethean cup, in spelling out their enchanting contents! Who would have them a whit less soiled? What better condition could we desire to see them in? 
 
而且,收藏这样的书,也不能给人带来什么不同凡响之感。说来也怪,由于这些作品本身如此脍炙人口,它们的外表如何并不能使书主感到高兴,也不能让他的占有欲得到什么额外的满足。我以为,汤姆逊的《四季》一书,样子以稍有破损、略带卷边儿为佳。对于一个真正爱读书的人来说,只要他没有因为爱洁成癖而把老交情抛在脑后,当他从“流通图书馆”借来一部旧的《汤姆?琼斯》或是《威克菲牧师传》的时候,那污损的书页、残破的封皮以及书上(除了俄罗斯皮以外)的气味,该是多么富有吸引力呀!它们表明了成百上千读者的拇指曾经带着喜悦的心情翻弄过这些书页,表明了这本书曾经给某个孤独的缝衣女工带来快乐。这位缝衣女工、女帽工或者女装裁缝,在干了长长的一天针线活之后,到了深夜,为了把自己的一肚子哀愁暂时浸入忘川之水,好不容易挤出个把钟头的睡眠时间,一个字一个字拼读出这本书里的迷人的故事。在这种情况之下,谁还去苛求这些书页是否干干净净、一尘不染呢?难道我们还会希望书的外表更为完美无缺吗?


%
%	In some respects the better a book is, the less it demands from binding. Fielding, Smollett, Sterne, and all that class of perpetually self-reproductive volumes—Great Nature's Stereotypes—we see them individually perish with less regret, because we know the copies of them to be “eterne.”But where a book is at once both good and rare—where the individual is almost the species, and when that perishes,
%
%We know not where is that Promethean torch.
%
%That can its light relumine—
%
%such a book, for instance, as the Life of the Duke of Newcastle, by his Duchess—no casket is rich enough, no casing sufficiently durable, to honour and keep safe such a jewel.
%
%…
%
%Milton almost requires a solemn service of music to be played before you enter upon him. But he brings his music, to which, who listens, had need bring docile thoughts, and purged ears.
%
%Winter evenings—the world shut out—with less of ceremony the gentle Shakespeare enters. At such a season, the Tempest, or his own Winter's Tale—
%
%These two poets you cannot avoid reading aloud—to yourself, or (as it chances) to some single person listening. More than one—and it degenerates into an audience.
%
%Books of quick interest, that hurry on for incidents, are for the eye to glide over only. It will not do to read them out. I could never listen to even the better kind of modern novels without extreme irksomeness.
%
%A newspaper, read out, is intolerable. In some of the Bank offices it is the custom (to save so much individual time) for one of the clerks—who is the best scholar—to commence upon the Times, or the Chronicle, and recite its entire contents aloud pro bono publico. With every advantage of lungs and elocution, the effect is singularly vapid. In barbers' shops and public-houses a fellow will get up, and spell out a paragraph, which he communicates as some discovery. Another follows with his selection. So the entire journal transpires at length by piece-meal. Seldom-readers are slow readers, and, without this expedient no one in the company would probably ever travel through the contents of a whole paper.
%
%Newspapers always excite curiosity. No one ever lays one down without a feeling of disappointment.
%
%What an eternal time that gentleman in black, at Nando's, keeps the paper! I am sick of hearing the waiter bawling out incessantly, “the Chronicle is in hand, Sir.”
%
%Coming in to an inn at night—having ordered your supper—what can be more delightful than to find lying in the window-seat, left there time out of mind by the carelessness of some former guest—two or three numbers of the old Town and Country Magazine, with its amusing tête-à-tête pictures—“The Royal Lover and Lady G—;”“The Melting Platonic and the Old Beau,”—and such like antiquated scandal? Would you exchange it—at that time, and in that place—for a better book?
%
%Poor Tobin, who latterly fell blind, did not regret it so much for the weightier kinds of reading—the Paradise Lost, or Comus, he could have read to him—but he missed the pleasure of skimming over with his own eye—a magazine, or a light pamphlet.
%
%I should not care to be caught in the serious avenues of some cathedral alone, and reading Candide!
%
%I do not remember a more whimsical surprise than having been once detected—by a familiar damsel—reclined at my ease upon the grass, on Primrose Hill, reading—Pamela. There was nothing in the book to make a man seriously ashamed at the exposure; but as she seated herself down by me, and seemed determined to read in company, I could have wished it had been — any other book. We read on very sociably for a few pages; and, not finding the author much to her taste, she got up, and—went away. Gentle casuist, I leave it to thee to conjecture, whether the blush (for there was one between us) was the property of the nymph or the swain in this dilemma. From me you shall never get the secret.
%
%I am not much a friend to out-of-doors reading. I cannot settle my spirits to it. I knew a Unitarian minister, who was generally to be seen upon Snow-hill (as yet Skinner's-street was not), between the hours of ten and eleven in the morning, studying a volume of Lardner. I own this to have been a strain of abstraction beyond my reach. I used to admire how he sidled along, keeping clear of secular contacts. An illiterate encounter with a porter's knot, or a bread basket, would have quickly put to flight all the theology I am master of, and have left me worse than indifferent to the five points.
%
%There is a class of street-readers, whom I can never contemplate without affection—the poor gentry, who, not having wherewithal to buy or hire a book, filch a little learning at the open stalls—the owner, with his hard eye, casting envious looks at them all the while, and thinking when they will have done. Venturing tenderly, page after page, expecting every moment when he shall interpose his interdict, and yet unable to deny themselves the gratification, they “snatch a fearful joy.”Martin B.—, in this way, by daily fragments, got through two volumes of Clarissa, when the stall-keeper damped his laudable ambition, by asking him (it was in his younger days) whether he meant to purchase the work. M. declares, that under no circumstances of his life did he ever peruse a book with half the satisfaction which he took in those uneasy snatches. A quaint poetess of our day has moralised upon this subject in two very touching but homely stanzas.
%
%I saw a boy with eager eye
%
%Open a book upon a stall,
%
%And read, as he'd devour it all;
%
%Which when the stall-man did espy,
%
%Soon to the boy I heard him call,
%
%“You, Sir, you never buy a book,
%
%Therefore in one you shall not look.”
%
%The boy pass'd slowly on, and with a sigh
%
%He wish'd he never had been taught to read,
%
%Then of the old churl's books he should have had no need.
%
%Of sufferings the poor have many,
%
%Which never can the rich annoy:
%
%I soon perceiv'd another boy,
%
%Who look'd as if he had not any
%
%Food, for that day at least—enjoy
%
%The sight of cold meat in a tavern larder.
%
%This boy's case, then thought I, is surely harder,
%
%Thus hungry, longing, thus without a penny,
%
%Beholding choice of dainty-dressed meat:
%
%No wonder if he wish he ne'er had learn'd to eat. 

}