 \section{八月一日}
\dailyquote{Only the soldier is a free man because he can look death in the face.
	
只有士兵才是一个自由的人,因为他能够直面死亡。}{ Friedrich Schiller(弗里德里希·席勒)}{mountainblow}

\poems{塞下曲(A Frontier Melody Li Bai)}{ 李白}{0.3\linewidth}{	五月天山雪,
	
	无花只有寒。
	
	
	笛中闻折柳,
	
	
	春色未曾看。
	
	
	晓战随金鼓,
	
	
	宵眠抱玉鞍。 
	
	
	愿将腰下剑,
	
	
	直为斩楼兰。
}{\0.7\linewidth}{
	The snows in Tianshan in the fifth moon are not yet to melt; 
	
	No flowers can be seen; howe'er, a bitter cold is felt. 
	
	The tune of Willow Twigs is often struck up on the flute, 
	
	But not an actual sign for spring has anywhere been spelt.
	
	By day, directed by the drum and gong men charge and fight;
	
	 With saddles grasp'd in arms they sleep with vigilance at night. 
	 
	 I wish, with the sharpen'd sword which I'm wearing on the waist, 
	 
	 To capture Loulan and put our foes in a fatal plight.}
\words{For Whom the Bell Tolls Ernest Hemingway(	丧钟为谁而鸣 )}{海明威}{
	He lay flat on the brown, pine-needled floor of the forest, his chin on his folded arms, and high overhead the wind blew in the tops of the pine trees. The mountainside sloped gently where he lay; but below it was steep and he could see the dark of the oiled road winding through the pass. There was a stream alongside the road and far down the pass he saw a mill beside the stream and the falling water of the dam, white in the summer sunlight.
	
		他匍匐在树林里褐色的、积着一层松针的地上,交叉的手臂支着下颚;在高高的上空,风在松树梢上呼啸而过。他俯躺着的山坡坡度不大,再往下却很陡峭,他看得到黑色的柏油路蜿蜒穿过山口。沿柏油路有条小河,山口远处的河边有家锯木厂,拦水坝的泄水在夏天的阳光下泛着白光。
	
	
	"Is that the mill?" he asked. "Yes." "I do not remember it." "It was built since you were here. The old mill is farther down; much below the pass."
	He spread the photostated military map out on the forest floor and looked at it carefully. The old man looked over his shoulder. He was a short and solid old man in a black peasant's smock and gray iron-stiff trousers and he wore rope-soled shoes. He was breathing heavily from the climb and his hand rested on one of the two heavy packs they had been carrying.
	
		“那就是锯木厂么?”他问。“就是。” “我记不得了。”“那是你离开这儿以后造的。老锯木厂还在前面,离山口很远。” 他在地上摊开影印的军用地图,仔细端详。老头儿从他肩后看着。他是个结实的矮老头儿,身穿农民的黑罩衣和硬邦邦的灰裤子,脚上是一双绳底鞋。他爬山刚停下来,还在喘气,一手搁在他们带来的两只沉重的背包的一只上面。 
	
	"Then you cannot see the bridge from here." "No," the old man said. "This is the easy country of the pass where the stream flows gently. Below, where the road turns out of sight in the trees, it drops suddenly and there is a steep gorge--" "I remember." "Across this gorge is the bridge." "And where are their posts?" "There is a post at the mill that you see there."
	
	“这么说从这里是望不到那座桥了。”“望不到,”老头儿说。“这山口一带地势比较平坦,水流不急。再往前,公路拐进林子不见了踪影,那里地势突然低下去,有个挺深的峡谷---”“我记得。”“峡谷上面就是那座桥。”“他们的哨所在哪儿?”“你看到的锯木厂那边有个哨所。”
	
	The young man, who was studying the country, took his glasses from the pocket of his faded, khaki flannel shirt, wiped the lenses with a handkerchief, screwed the eyepieces around until the boards of the mill showed suddenly clearly and he saw the wooden bench beside the door; the huge pile of sawdust that rose behind the open shed where the circular saw was, and a stretch of the flume that brought the logs down from the mountainside on the other bank of the stream. The stream showed clear and smooth-looking in the glasses and, below the curl of the falling water, the spray from the dam was blowing in the wind.
	
	 这个正在研究地形的年轻人从他褐色的黄褐色法兰绒衬衫口袋里掏出望远镜,用手帕擦擦镜片,调整焦距,目镜中的景象突然清晰,连锯木厂的木板都看到了,他还看到了门边的一条长板凳,敞棚里的圆锯,后面有一大堆木屑;他还看到一段把小河对岸山坡上的木材运下来的滑槽。小河在望远镜里显得清澈而平静,打着漩涡的水从拦水坝泻下来,底下的水花在风中飞溅。
	 
	"There is no sentry." "There is smoke coming from the millhouse," the old man said. "There are also clothes hanging on a line." "I see them but I do not see any sentry." "Perhaps he is in the shade," the old man explained. "It is hot there now. He would be in the shadow at the end we do not see." "Probably. Where is the next post?" "Below the bridge. It is at the roadmender's hut at kilometer five from the top of the pass." "How many men are here?" He pointed at the mill.
	 "Perhaps four and a corporal." "And below?" "More. I will find out." "And at the bridge?" "Always two. One at each end." "We will need a certain number of men," he said. "How many men can you get?" "I can bring as many men as you wish," the old man said."There are many men now here in the hills." "How many?" "There are more than a hundred. But they are in small bands. How many men will you need?" "I will let you know when we have studied the bridge." "Do you wish to study it now?" "No. Now I wish to go to where we will hide this explosive until it is time. I would like to have it hidden in utmost security at a distance no greater than half an hour from the bridge, if that is possible." "That is simple," the old man said. "From where we are going, it will all be downhill to the bridge. But now we must climb a little in seriousness to get there. Are you hungry?" "Yes," the young man said. "But we will eat later. How are you called? I have forgotten." It was a bad sign to him that he had forgotten. "Anselmo," the old man said. "I am called Anselmo and I come from Barco de Avila. Let me help you with that pack."
	 
	  “没有岗哨。”“锯木房里在冒烟,”老头儿说。“还有晒衣绳上挂着衣服。”“这些我见到了,但看不到岗哨。”“说不定他在背阴处,”老头儿解释说。“那儿现在挺热。他也许在我们看不到的背阴那头。”“可能。另一个哨所在哪里?”“在桥下方。在养路工的小屋边,离山口五公里的里程碑那里。”“这里有多少士兵?”他指指锯木厂。“也许有四个加上一个班长。”“下面呢?”“要多些。我能探听明白。”“那么桥头呢?”“总是两个。每边一个。”“我们需要一批人手,”他说。“你能召集多少?”“你要多少,我就能召集多少,”老头儿说。“这一带山里现在就有不少人。”“多少?”“一百多个。不过他们三三五五分散开了。你需要多少人?”“等我们勘察了桥以后再跟你说。”“你想现在就去勘察桥吗?”“不。现在我想去找个地方把炸药藏起来,要用的时候再去取。我希望把它藏在最安全的地方,假如可能的话,离桥不能超过半个小时的路程。”“那简单,”老头儿说。“从我们现在要去的地方到桥头全都是下坡路。不过,我们现在要去那儿倒得很认真地爬一会山哪。你饿吗?”“饿,”年轻人说。“不过,我们过后再吃吧。你叫什么名字?我忘了。”他竟把名字都忘了,这对他来说是个不祥之兆。“安塞尔莫,”老头儿说。“我叫安塞尔莫,老家在阿维拉省的巴尔科城。我来帮你拿那只背包。” 
	  
	The young man, who was tall and thin, with sun-streaked fair hair, and a wind- and sun-burned face, who wore the sun-faded flannel shirt, a pair of peasant's trousers and rope-soled shoes, leaned over, put his arm through one of the leather pack straps and swung the heavy pack up onto his shoulders. He worked his arm through the other strap and settled the weight of the pack against his back. His shirt was still wet from where the pack had rested.
	
	这年轻人是个瘦高个儿,张着闪亮的金发和一张饱经风霜日晒的脸,他穿着一件晒得褪了色的法兰绒衬衫,一条农民的裤子和一双绳底鞋。他弯下腰去,一条胳膊伸进背包皮带圈里,把那沉重的背包甩上肩头。他把另一条胳膊伸进另一条皮带圈里,使背包的重量全压在背上。他衬衫上原先被背包压住的地方还是汗湿的。 
	
	"I have it up now," he said. "How do we go?" "We climb," Anselmo said.
	
	“我把它背上啦,”他说。“我们怎么走?”“咱俩爬山。”安塞尔莫说。
	
	Bending under the weight of the packs, sweating, they climbed steadily in the pine forest that covered the mountainside. There was no trail that the young man could see, but they were working up and around the face of the mountain and now they crossed a small stream and the old man went steadily on ahead up the edge of the rocky stream bed. The climbing now was steeper and more difficult, until finally the stream seemed to drop down over the edge of a smooth granite ledge that rose above them and the old man waited at the foot of the ledge for the young man to come up to him.
	
	他们被背包压得弯下了腰,在山坡上的松树林里一步步向上爬,身上淌着汗。年轻人发现林中并没有路径,但是他们继续向上攀登,绕到了前山,这时跨过了一条小溪,老头儿踩着溪边石块稳健地向前走去。这时,山路更陡峭,爬山更艰难了,到后来,溪水似乎是从他们头顶上一个平滑的花岗石悬崖边上直泻下来,于是老头儿在悬崖下停了步,等着年轻人赶上来。
	
	"How are you making it?" "All right," the young man said. He was sweating heavily and his thigh muscles were twitchy from the steepness of the climb. "Wait here now for me. I go ahead to warn them. You do not want to be shot at carrying that stuff." "Not even in a joke," the young man said. "Is it far?" "It is very close. How do they call thee?" "Roberto," the young man answered. He had slipped the pack off and lowered it gently down between two boulders by the stream bed. "Wait here, then, Roberto, and I will return for you." "Good," the young man said. "But do you plan to go down this way to the bridge?" "No. When we go to the bridge it will be by another way. Shorter and easier." "I do not want this material to be stored too far from the bridge." "You will see. If you are not satisfied, we will take another place." "We will see," the young man said. He sat by the packs and watched the old man climb the ledge. It was not hard to climb and from the way he found hand-holds without searching for them the young man could see that he had climbed it many times before. Yet whoever was above had been very careful not to leave any trail.
	
	“你行吗?” “行,”年轻人说。他大汗淋漓,因为爬了陡峭的山路,大腿的肌肉抽搐起来。 “在这里等我。我先去通知他们。你带了这玩意总不希望人家朝你开枪吧。” “当然不希望,”年轻人说。“路远吗?” “很近。怎么称呼你?” “罗伯托(这是本书主人公罗伯托 乔丹的名字的西班牙语读法的音译。),”年轻人回答。他卸下背包,轻轻地放在溪边两块大圆石之间。 “那么就在这儿等着,罗伯托,我回来接你。” “好,”年轻人说。“难道你打算以后走这条路到下面桥头吗?” “不。我们到桥头去得走另一条路。那条路近一些,比较容易走。” “我不想把这东西藏得离桥太远。” “你瞧着办吧。要是你不满意,我们另找地方。” “我们瞧着办吧,”年轻人说。	他坐在背包旁边,看着老头儿攀登悬崖。这悬崖不难攀登,而且这年轻人发现,从老头儿不用摸索就找到攀手地方的利落样子看来,这地方他已经爬过好多次了。然而,待在上面的人们一向小心翼翼地不让留下任何痕迹来。}