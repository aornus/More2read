 \section{八月三日}
 \dailyquote{Time, whose tooth gnaws away everything else, is powerless against truth. 
	
	时间的利齿可以吞噬一切别的东西,对真理却无能为力。}{Thomas Hexley(托马斯·赫胥黎)}{七夕}
 \poems{迢迢牵牛星(Parted Lovers)}{古诗十九首}
 {0.3\linewidth}
 {迢迢牵牛星,
 	
 	皎皎河汉女。
 	
 	 纤纤擢素手,
 	 
 	 札札弄机杼。 
 	 
 	 终日不成章,
 	 
 	 泣涕零如雨。
 	 
 	  河汉清且浅,
 	  
 	  相去复几许。 
 	  
 	  盈盈一水间,
 	  
 	  脉脉不得语。}
 {0.7\linewidth}
 { 	
 	Far, far away, the Cowherd, 
 	
 	Fair, fair, the Weaving Maid;
 	
 	 Nimbly move her slender white fingers, 
 	 
 	 Click-clack goes her weaving-loom.
 	 
 	 
 	All day she weaves, yet her web is still not done.
 	
 	 And her tears fall like rain. 
 	 
 	 Clear and shallow the Milky Way, 
 	 
 	 They are not far apart! 
 	 
 	 But the stream brims always between. 
 	 
 	 And, gazing at each other, they cannot speak.}
 \words{ 埃利斯·贝尔与阿克顿·贝尔生平纪略
}{ 	夏洛蒂·勃朗特(刘炳善 译)}{ 
\begin{center}
	Biographical Notice of Ellis and Acton Bell
\end{center}
It has been thought that all the works published under the names of Currer, Ellis, and Acton Bell were, in reality, the production of one person. This mistake I endeavoured to rectify by a few words of disclaimer prefixed to the third edition of ‘Jane Eyre.’

These, too, it appears, failed to gain general credence, and now, on the occasion of a reprint of ‘Wuthering Heights’ and ‘Agnes Grey,’ I am advised distinctly to state how the case really stands.

Indeed, I feel myself that it is time the obscurity attending those two names — Ellis and Acton — was done away. The little mystery, which formerly yielded some harmless pleasure, has lost its interest; circumstances are changed. It becomes, then, my duty to explain briefly the origin and authorship of the books written by Currer, Ellis, and Acton Bell.

About five years ago, my two sisters and myself, after a somewhat prolonged period of separation, found ourselves reunited, and at home.

Resident in a remote district, where education had made little progress, and where, consequently, there was no inducement to seek social intercourse beyond our own domestic circle, we were wholly dependent on ourselves and each other, on books and study, for the enjoyments and occupations of life.

The highest stimulus, as well as the liveliest pleasure we had known from childhood upwards, lay in attempts at literary composition; formerly we used to show each other what we wrote, but of late years this habit of communication and consultation had been discontinued; hence it ensued, that we were mutually ignorant of the progress we might respectively have made.

One day, in the autumn of 1845, I accidentally lighted on a MS. volume of verse in my sister Emily’s handwriting. Of course, I was not surprised, knowing that she could and did write verse: I looked it over, and something more than surprise seized me — a deep conviction that these were not common effusions, nor at all like the poetry women generally write.

I thought them condensed and terse, vigorous and genuine. To my ear they had also a peculiar music — wild, melancholy, and elevating.

My sister Emily was not a person of demonstrative character, nor one on the recesses of whose mind and feelings even those nearest and dearest to her could, with impunity, intrude unlicensed; it took hours to reconcile her to the discovery I had made, and days to persuade her that such poems merited publication.

I knew, however, that a mind like hers could not be without some latent spark of honourable ambition, and refused to be discouraged in my attempts to fan that spark to flame.

Meantime, my younger sister quietly produced some of her own compositions, intimating that, since Emily’s had given me pleasure, I might like to look at hers.I could not but be a partial judge, yet I thought that these verses, too, had a sweet, sincere pathos of their own.

We had very early cherished the dream of one day becoming authors. This dream, never relinquished even when distance divided and absorbing tasks occupied us, now suddenly acquired strength and consistency: it took the character of a resolve. We agreed to arrange a small selection of our poems, and, if possible, to get them printed.

Averse to personal publicity, we veiled our own names under those of Currer, Ellis, and Acton Bell; the ambiguous choice being dictated by a sort of conscientious scruple at assuming Christian names positively masculine, while we did not like to declare ourselves women, because — without at that time suspecting that our mode of writing and thinking was not what is called ‘feminine’ — we had a vague impression that authoresses are liable to be looked on with prejudice …

… We had noticed how critics sometimes use for their chastisement the weapon of personality, and for their reward, a flattery, which is not true praise.

The bringing out of our little book was hard work. As was to be expected, neither we nor our poems were at all wanted; but for this we had been prepared at the outset; though inexperienced ourselves, we had read the experience of others.

The great puzzle lay in the difficulty of getting answers of any kind from the publishers to whom we applied. Being greatly harassed by this obstacle, I ventured to apply to the Messrs. Chambers, of Edinburgh, for a word of advice; they may have forgotten the circumstance, but I have not, for from them I received a brief and business—like, but civil and sensible reply, on which we acted, and at last made a way.

The book was printed: it is scarcely known, and all of it that merits to be known are the poems of Ellis Bell. The fixed conviction I held, and hold, of the worth of these poems has not indeed received the confirmation of much favourable criticism; but I must retain it notwithstanding.

Ill-success failed to crush us: the mere effort to succeed had given a wonderful zest to existence; it must be pursued. We each set to work on a prose tale: Ellis Bell produced ‘Wuthering Heights,’ Acton Bell ‘Agnes Grey,’ and Currer Bell also wrote a narrative in one volume.

These MSS. were perseveringly obtruded upon various publishers for the space of a year and a half; usually, their fate was an ignominious and abrupt dismissal.

At last ‘Wuthering Heights’ and ‘Agnes Grey’ were accepted on terms somewhat impoverishing to the two authors; Currer Bell’s book found acceptance nowhere, nor any acknowledgment of merit, so that something like the chill of despair began to invade her heart. As a forlorn hope, she tried one publishing house more — Messrs. Smith, Elder and Co.

Ere long, in a much shorter space than that on which experience had taught her to calculate — there came a letter, which she opened in the dreary expectation of finding two hard, hopeless lines, intimating that Messrs. Smith, Elder and Co. ‘were not disposed to publish the MS.,’ and, instead, she took out of the envelope a letter of two pages.

She read it trembling. It declined, indeed, to publish that tale, for business reasons, but it discussed its merits and demerits so courteously, so considerately, in a spirit so rational, with a discrimination so enlightened, that this very refusal cheered the author better than a vulgarly expressed acceptance would have done.

It was added, that a work in three volumes would meet with careful attention.

\\

长期以来,在柯勒、埃利斯和阿克顿·贝尔的署名下所发表的作品,一直被认为统统不过是某一个人的化名之作。对此误解,我曾在《简·爱》第三版书前以寥寥数语予以否认和纠正,但那番话看来并未得到大家相信。所以,当此《呼啸山庄》重印之际,我接受建议,愿将事实真相加以澄清。

而且,我个人也深深感到:笼罩着埃利斯和阿克顿这两个名字的迷茫之雾,现在确实应该驱散了。那种小小的秘密,往日曾给我们一点点善良无害的快乐,由于时过境迁,早已失去了原来的兴味。今天,我责无旁贷,理应对于柯勒、埃利斯和阿克顿·贝尔所写各书的来历和著作权,加以简短说明。

约当五年以前,我的两个妹妹和我,在相当长时期的分别之后,又在家中重新会面。住在偏远之地,教育素不发达,故于亲人团聚以外,殊乏拜客访友之趣;日常心之所乐、情之所寄,唯有姊妹间相亲相依,唯有读书一事而已。好在我们自孩童时代以来所极感振奋、乐此不疲之事尚有文学习作。往日我们常将自己作品互相传阅,但后来几年此种交流、磋商业已中断,因而姊妹间对于各自写作进展情况不免隔膜。

1845年秋季的一天,我偶尔看到二妹艾米莉手写的一卷诗稿。当然,对此我并不觉得奇怪,因为我知道她赋有诗才且不断写诗。然而披览之后,我仍不禁深为震惊,感到这些诗歌绝非平平之作。它们毫无通常所谓的脂粉气息,而是精炼、简洁、刚健、率真。在我耳中,这些诗歌具有一种特殊的音韵之美--它们粗犷、忧郁、崇高。

艾米莉生性含而不露。埋藏在她心底的感情秘密,虽是至亲至近之人,非经许可也不得贸然侵犯。因此,仅仅诗稿被我发现一事,就需我解释几个小时,她才释然于怀;而使她相信这些诗歌确有发表价值,又费我整整几天。然而我认为,像她那样性格的人,在内心深处绝不会没有潜伏着远大抱负的星星之火;不把这星星之火煽成熊熊火焰,我决不罢休。

与此同时,我的小妹也悄悄拿出了她的创作,并且吐露说:既然我对艾米莉的作品感到高兴,或许对她的作品也肯一顾。要我来对这些诗歌下个断语,恐怕不免有偏爱之嫌,然而我还是要说,她的这些诗也具有自己真挚可爱的凄婉情趣。

我们姊妹早在幼小时候就抱着有朝一日成为作家的梦想。后来虽则三人天各一方,且又重务缠身,但此心此志从未抛却;如今一旦重新获得力量,便分外坚定,并形成决心。我们决定编选一本小小的诗集,并尽可能将其出版。不想把自己身份公之于众,我们采用了柯勒·贝尔、埃利斯·贝尔和阿克顿·贝尔的假名,将自己真名隐去;而选取这种模棱两可的名字,乃由于一方面不愿公开自己的女性身份,同时出于谨慎的顾虑,也不愿采用那些一望而知即是男性的名字。其所以如此,又是因为——尽管我们自知自己的笔法和思路并无一般所谓的“女儿气”——我们有一种笼统印象,就是:人们看待女作家往往怀着偏见,批评家有时拿性别当作惩罚的武器,有时又以此作为吹捧的因由——而吹捧当然不是真实的赞扬。

我们这本小书,出版实非易事。正如事前所料,不论我们这三个作者或是我们的诗歌,都不受人欢迎。不过,对此我们早有准备,因为我们自己虽是生手,却也读过他人的甘苦之谈。最使我们困惑不解的莫过于向出版商提出的请求都音信杳然。为此烦困之余,我只得向爱丁堡的钱伯斯公司诸先生冒昧投书,讨个主意。对于此事,他们或已忘在脑后,我却记忆犹新,因为从他们那里我收到了一个短短的、事务性的,同时也是有礼貌的、切切实实的答复。我们遵嘱而行,出书的事才算有了眉目。

诗集出来了,但知音寥寥,而其中确值得为人所知的作品乃是埃利斯·贝尔的诗歌——对于这些诗的价值,我过去、现在都确信不疑;尽管此种信念尚未得到批评界的认可,我却坚持不变。

失败没有压垮我们,仅仅为了成功而奋斗本身就给人生以极大乐趣。一定要坚持下去。我们每人动手写一部小说:埃利斯·贝尔写了《呼啸山庄》,阿克顿·贝尔写了《阿格尼丝·格雷》,柯勒·贝尔也写了一部一卷本的作品。这三部稿子,在一年半当中接连闯入一家又一家出版社——它们所遭受的命运往往是在寄出不久就又灰溜溜地给退回来了。

最后,《呼啸山庄》和《阿格尼丝·格雷》被人接受了,但出版条件对两位作者相当苛刻。柯勒·贝尔的书仍然到处碰壁,无人赏识。绝望,犹如一股寒流,侵袭她的内心。作为无望中之希望,她把稿子寄给另一家出版社——老史密斯公司。不久,比她根据以往经验所估计的时间要快得多,回信来了。她无精打采地把信拆开,预料内容不过是两行冷冰冰、毫无希望的字句,通知说老史密斯公司“对大作不拟刊用”,然而这次她却从信封里拿出两页信纸。她捧读时不禁心悸手颤。信中说鉴于营业上的原因,公司不打算出版此书;但接着信里分析了稿子的优点和缺点,措辞如此礼貌,考虑如此周到,态度如此合理,识见如此通达,这样的退稿真比粗俗的采纳更使作者感到快慰。信里还说若能有一部三卷本的作品,将会受到重视。
}