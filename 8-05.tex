 \section{八月五日}
 \dailyquote{I have ever wasted time, but now time expends me. 
 	
 	我曾经浪费过时间,现在时间开始消耗我。}{William Shakespeare(威廉·莎士比亚)}{槐花}
 \poems{Life in a Love(爱的生活) }{Robert Browning(罗伯特·布朗宁,杨中仁 译)}{0.5\linewidth}{ 
 	Escape me? 
 	
 	Never- 
 	
 	Beloved! 
 	
 	While l am l, and you are you, 
 	
 	So long as the world contains us both, 
 	
 	Me the loving and you the loth, 
 	
 	While the one eludes, must the other pursue. 
 	
 	My life is a fault at last, I fear: 
 	
 	It seems too much like a fate, indeed! 
 	
 	Though I do my best l shall scarce succeed 
 	
 	But what if l fail of my purpose here?
 	\\
 	
 	lt is but to keep the nerves at strain, 
 	
 	To dry one's eyes and laugh at a fall, 
 	
 	And baffled, get up to begin again,- 
 	
 	So the chase takes up one's life, that's all. 
 	
 	While, look but once from your farthest bound 
 	
 	At me so deep in the dust and dark, 
 	
 	No sooner the old hope drops to ground 
 	
 	Than a new one, straight to the selfsame mark, 
 	
 	I shape me- 
 	
 	Ever 
 	
 	Removed!
}{0.5\linewidth}{
 	心爱的! 

你绝不能 

弃我而去啊! 

我是我,你还是你, 

只要世界里还有你我, 

我爱着你,你却不爱我, 

我追求你,而你却躲着我。

恐怕我这一生就是个错, 

其实,我知道天命难违! 

就算费尽心血,我的机会也不多。 

实现不了初衷我该怎么活?
\\

那我的神经更会紧张哆嗦, 

擦干眼泪,昂首笑对起起落落,

如果受挫,爬起来再做,

追求要执著,仅此而已。

然而,你从最远处看一看 

深陷在黑暗尘埃中的我,

前面的希望若落空了 

新希望再攻旧目标, 

我塑造的我, 

任何时间 

不会变!
}
 \words{槐花吟 (An Ode to the Acacia Flower)}{周领顺}{


 	
 	槐花既可赏,也能食,对北方人而言,想到更多的是食用:蒸槐花、炕槐花饼、包槐花包子,应有尽有,你只要想得到,就能做得出,纯粹是大自然的尤物,不过十来日,便随风而去。所以,吃槐花,吃的就是个时鲜,而在这青黄不接的春季里,竟也能有秋的收获。
 	
 	槐花默认的是洋槐树上结的洋槐花。称之为洋槐,是为外来物种之故,19世纪下半叶才从北美传入中国,所以白居易《秋日》里描绘的“袅袅秋风多,槐花半成实”和子兰《长安早秋》里描绘的“风舞槐花落御沟,终南山色入城秋”,只能是对于秋季里国槐花的描摹。国槐常做景观树,树冠如伞,丝绦垂地,但随处可见的却是洋槐,农家庭院和沟边水泽不乏它的身影,除了极易成活的原因外,凡有意栽种者,必定因其材质好,所做家具耐用,绝非仅为一年到头这十来日的花食。但槐花含有丰富的蛋白质、脂肪酸、维生素和矿物质,具有降血压和扩张冠状动脉等功效,集食用和药用于一身。当然,这些知识若非专门查证,一般人断难知其详。
 	
 	每到花期来临,白中泛绿的槐花,反射着玉的质地;一串串缀满枝条,空气中弥漫着素雅的清香。槐花飘香时,盛春已然至。
 	
 	清香,准确地说并不是近身的感觉,如果置身槐花丛中,就只有用浓香状述其味了,不仅香,而且香得呛人。浓香随风转至清香,招揽了蜜蜂,所以有了槐花蜜;招揽了行人,所以有了槐花痴,而更有过之的,当然是青睐槐花食的男男女女。成语有“秀色可餐”之说,以秀色代餐,使人忘掉了饥饿,但槐花却能令人陡增食欲,蒸槐花的蒜香,炕槐花饼的焦香,槐花包子的素香,一古脑就都来了,画面感十足,让流连者不仅赏之,甚或烹而食之。
 	
 	采摘槐花是有讲究的,既要特别提防树枝上的木刺,又要看准花的形态。槐树有刺,分布于枝叶间,又硬又尖,采摘一回槐花,要是手不被扎个三两点,就算得上采摘老手。虽然槐树皮粗糙,适合少年攀爬,但因木刺当道,树梢上的鸟巢便总能幸免于难。槐花从出生到完全成熟,大概有三种形态:初成米粒状,虽可食用,但有青涩感;接近微黄时,已垂垂老矣。最好是花苞,呈月牙样,吃起来香喷喷、甜丝丝。
 	
 	米粒状的槐花,尚不具备花的形态,采摘下来,委实可惜。不妨留于枝头,待吃上几天的花苞,那些槐米也就到了采摘的最佳期。而成熟的老槐花除了颜色泛黄、形状怒放可以辨识外,轻轻一抖,还会有花瓣飘落,留下线状的花蕊在花萼里抖动,骨感十足。老槐花并非不能食用,晒干后包包子,口感劲道,所以过去常有老年人把大风吹落的干槐花扫拢备用,只是在当季尝鲜时,必以花苞为上品。
 	
 	南方人少知可食之树花,大概只有桂花、木棉花等少数几种。作为北方人,我不仅知道槐花能够食用,还品尝过榆树上的榆钱、构树上的蒲穗、泡桐树上的桐花、柳树上的柳絮,凡此种种,不仅好吃,且都有药用价值。北方的春季,总会涌动着采撷树上时鲜的人流,形成不绝如缕的流动风景。历史上南方比北方富裕,生活没把南方人逼到遍尝百草的地步,幸福如是,但也错失了品尝树花的口福。槐花之德,必吟之而后快。
 \\
 
 	
 	The acacia flower appeals in both its beauty and flavor. And, consuming the acacia flower is what occurs most often to Northern Chinese, typically in the form of steaming, baking, or stuffing. Whatever your imagination can lead you to, it can be served to your taste. The acacia flower is an absolutely special gift from nature, and within a lapse of ten days or so, it will be gone with a gust of wind. Therefore, acacia flowers are a seasonal delicacy. Alas, we have such a harvest as the fruitful autumn can bestow especially in the so-called fruitless spring.
 	
 	The acacia flower comes from the acacia tree with the prefix of yang (meaning foreign) in Chinese referring to something abroad. The acacia tree is from North America and was brought to China in the second half of the 19th century. As the acacia flower and the Chinese scholar tree flower both share the same head word huai, the poetic lines from poet Bai Juyi and those from poet Zi Lan in the Tang dynasty are in fact descriptions of the Chinese scholar tree flower in the autumn rather than the acacia flower in the spring. The Chinese scholar tree is often planted as a landscape tree with the crown of the tree taking the shape of an umbrella with its branches drooping. However, it is the acacia tree that comes into sight more often, being in the courtyard and next to the waters. Apart from the reason that the acacia tree is easy to grow, the tree’s delicate yet durable texture makes it a fine wood to craft furniture from, which means that it is not only grown purely for this ten-day flower feast, but also for other purposes. The acacia flower is quite rich in protein, fatty acids, vitamins and mineral substances with the function of reducing blood pressure and unclogging arteries, edible and medicinal being in one. Surely, ordinary people would not be so well-informed unless they purposely searched for the knowledge.
 	
 	With the blooming season quickly approaching, the white acacia flower with its greenish edges reflects the texture of jade. Strings of acacia flowers are abundant on branches and twigs, which fill the air with a faint scent. The fragrance of the acacia flower entails the prime time of spring.
 	
 	The so-called “faint scent” is far from faint. When surrounded by acacia flowers it is not only fragrant but fragrant enough to appear stifling. The aroma turns into a faint scent only in the wind, attracting bees, and with bees come honey; it also attracts passersby, hence plant-enthusiasts. Going even further are the men and women longing for the acacia flower feast. There is a Chinese idiom that says “Be beautiful enough to feast the eyes”, which means the flower is so beautiful that you can feast your eyes on it and forget about your real hunger. But on the contrary, the acacia flower can increase your appetite dramatically: steamed acacia flowers with garlic flavor, cake made from acacia flowers with a burnt odor and a steamed stuffed bun with plain fragrance, all come to mind at once; all these are graphic, contributing to the appreciation of visitors and attracting them to cook sometimes.
 	
 	To pluck the acacia flower, you should be particular about not only its thorns, but also its shape. The acacia tree has hard sharp thorns scattered all over its branches and leaves. You are a real seasoned picker if your hands haven’t been pricked two or three times while collecting the flowers. Though the acacia tree has a bark rough enough for teenagers to climb, the thorns provide a little protection and safety for nesting birds. The acacia flower has roughly three shapes from first shoot to full bloom. When it first appears to be the shape of a rice grain, it can be edible though tart. When the flower turns a yellowish color, it is already too ripe. The best time to eat the acacia flower is when it is in the bud, appearing like a crescent moon and tasting sweet and delicious.
 	
 	When the acacia flowers have the shape of a rice grain, it hasn’t quite taken the shape of a flower yet. It is quite a pity to pluck them this early and it is best to leave them to flower for a few days. You can eat the buds while waiting for the rice grains to be at their prime time to be plucked. Besides that, the petals of the fully ripe acacia flowers can be distinguished by their yellowish color and full blossoms; besides, the fully ripe acacia flowers fall with just the slightest touch, leaving a bare bony pistil shivering at the center of the sepals. Fully ripe acacia flowers can be eaten, too, especially when wrapped in a steamed baozi, the flavor being chewy. In the old days, old men and women used to round up the fallen dried flowers to save them for future use. But if you’d like to have a taste in season, the buds are of course the preferred choice.
 	
 	People from the South only know of a few edible flowers, such as sweet-scented osmanthus and common bomhax flowers. Being a northerner, I know and taste not only the acacia flower, but also the elm seeds on the elm tree, the panicles on the paper mulberry tree, the paulownia flowers on the paulownia tree and the willow catkins on the willow tree. Flowers like these are tasty and have medicinal values as well. In the springtime in the North, there are often streams of people plucking flowers off trees, which make up flows of scenery. In history, Southerners were richer than Northerners, so they need not taste all kinds of herbs to make them feel full. Though happy, they have missed opportunities to satisfy their appetite with edible tree flowers. The virtues of the acacia flower deserve an ode.
 }