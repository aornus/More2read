 \section{八月十日}
 \dailyquote{Procrastination is the art of keeping up with yesterday. 
 	
 	拖延是止步于昨日的艺术。
 }{Don Marquis(唐·马奎斯)}{Love}
 \poems{村姑 	}{戴望舒}{0.5\linewidth}{
  村里的姑娘静静地走着,
 
 提着她的蚀着青苔的水桶;
 
 溅出来的冷水滴在她的跣足上,
 
 而她的心是在泉边的柳树下。
 
 这姑娘会静静地走到她的旧屋去,
 
 那在一棵百年的东青树荫下的旧屋,
 
 而当她想到在泉边吻她的少年,
 
 她会微笑着,抿起了她的嘴唇。
 
 她将走到那古旧的木屋边,
 
 她将在那里惊散了一群在啄食的瓦雀,
 
 她将静静地走到厨房里,
 
 又静静地把水桶放在干草边。
 
 她将帮助她的母亲造饭,
 
 而从田间回来的父亲将坐在门槛上抽烟,
 
 她将给猪圈里的猪喂食,
 
 有将可爱的鸡赶进它们的窠里去。
 
 在暮色中吃晚饭的时候,
 
 她父亲会谈着今年的收成,
 
 她或许会说到她的女儿的婚嫁,
 
 而她便将羞怯地低下头去。
 
 她的母亲或许会说她的懒惰,
 
 (她打水的迟疑便是一个好例子,)
 
 但是她会不听到这些话,
 
 因为她在想着那有点鲁莽的少年。
 
}{0.5\linewidth}{The country girl she quietly tripped along

Carrying a bucket green with lichen,

Her feet were sprinkled by the splashing water,

Her heart was under the willow by the well.

The girl would quietly walk to her old cottage

Under the centenarian evergreen

But when she thought of the boy who had kissed her by the well,

She would smile and purse her lips.

Towards her cottage turning,

She would scare to flight a flock of pecking sparrows,

Quietly she would walk into the kitchen

And quietly drop the bucket by the hay.

She would help her mother to prepare the meal

And her father, back from the fields, would sit and smoke;

She would feed the pigs and drive the fowls to roost.

At dinner in the twilight

Her father would discourse on this year’s harvest,

Mutter some words anent his daughter’s marriage—

Then, timidly, the girl would bend her head.

Her mother would complain of her laziness

(That dallying by the well was an example)

But she never even heard her mother’s speech;

She was thinking the boy had been a little rough.

}
 \words{ A Letter to Thoreau (Excerpt)给梭罗的一封信
 }{	Edward O. Wilson  (爱德华·威尔逊(译/杨玉龄))
}{ I understand why you came to Walden Pond; your words are clear enough on that score. Granted, you chose this spot primarily to study nature. But you could have done that as easily and far more comfortably...
 
 Here is what I believe happened. You sought enlightenment and fulfillment the Old Testament way, by reduction of material existence to the fundamentals. When you stripped your outside obligations to the survivable minimum, you placed your trained and very active mind in an unendurable vacuum. And this is the essence of the matter: in order to fill the vacuum, you discovered the human proclivity to embrace the natural world.
 
 You searched for essence at Walden and, whether successful in your own mind or not, you hit upon an ethic with a solid feel to it: nature is ours to explore forever; it is our crucible and refuge; it is our natural home; it is all these things. Save it, you said: in wildness is the preservation of the world.
 
 Now, in closing this letter, I am forced to report bad news. (I put it off till the end.) The natural world in the year 2001 is everywhere disappearing before our eyes—cut to pieces, mowed down, plowed under, gobbled up, replaced by human artifacts.
 
 No one in your time could imagine a disaster of this magnitude. Little more than a billion people were alive in the 1840s. They were overwhelmingly agricultural, and few families needed more than two or three acres to survive. The American frontier was still wide open. And far away on continents to the south, up great rivers, beyond unclimbed mountain ranges, stretched unspoiled equatorial forests brimming with the maximum diversity of life.
 
 These wildernesses seemed as unattainable and timeless as the planets and stars. That could not last, because the mood of Western civilization is Abrahamic. The explorers and colonists were guided by a biblical prayer: May we take possession of this land that God has provided and let it drip milk and honey into our mouths, forever.
 
 The race is now on between the technoscientific forces that are destroying the living environment and those that can be harnessed to save it. We are inside a bottleneck of overpopulation and wasteful consumption. If the race is won, humanity can emerge in far better condition than when it entered, and with most of the diversity of life still intact.
 
 Henry, my friend, thank you for putting the first element of that ethic in place. Now it is up to us to summon a more encompassing wisdom. The living world is dying; the natural economy is crumbling beneath our busy feet. We have been too self-absorbed to foresee the long-term consequences of our actions, and we will suffer a terrible loss unless we shake off our delusions and move quickly to a solution. Science and technology led us into this bottleneck. Now science and technology must help us find our way through and out.
 
 You once said that old deeds are for old people, and new deeds are for new. I think that in historical perspective it is the other way around. You were the new and we are the old. Can we now be the wiser? For you, here at Walden Pond, the lamentation of the mourning dove and the green frog’s t-r-r-oonk! across the predawn water were the true reason for saving this place. For us, it is an exact knowledge of what that truth is, all that it implies, and how to employ it to best effect. So, two truths. We will have them both, you and I and all those now and forever to come who accept the stewardship of nature.
\\

 我了解你为什么要到瓦尔登湖畔来居住,对此,你说得够明白了。没错儿,你选择这个地点为的是研究大自然。但是你大可更轻松地去观察大自然……
 
 以下是我的推论。你渴慕神灵,因此你试图把物质生活降到最基本的水平,以寻求事物的真谛以及《旧约圣经》的实践之道。当你将身外的牵绊降低到最少时,你那训练有素且敏锐的心灵,顿时落入无法忍受的真空之中。而这就是事物的本质:为了要填补这份真空,你发现了人类的天性——拥抱大自然。
 
 你来到瓦尔登湖寻求人生精义,不论在你心里认为是否成功,你都谈到了一项感触很深的道理:大自然永远能供我们探索,它既是对我们的考验,也是我们的避难所,它是我们天生的家园,它就是一切。救救它吧,你说过,保护世界就在于保护它的野性。
 
 这封信写到尾声,现在,我不得不报告坏消息了。(我拖到最后再说。)2001年,大自然在你我眼前随处消失——被切碎、摧毁、犁耕、攫取、取代,这一切都是人类所为。
 
 你那个时代的人,恐怕想象不出规模这等宏大的破坏。1840年代,地球人口只有10亿多一些。他们绝大多数以务农为生,少数人家只需要两三英亩的土地就可以生活。当时美国境内还有很辽阔的土地未开垦。美国以南的几块大陆上,那些大河流域上游、难以攀越的高山上,长满未经破坏的热带雨林,里面的生物多样性丰富至极。
 
 当时这些野生生物仿佛天上的星辰难以企及,永远存在。但是由于西方文明的情感是亚伯拉罕式的,这种情况注定不会长久。探险家和殖民者遵守的都是《圣经》里的祈祷:让我们拥有上帝所赐给我们的流淌着奶与蜜的美地,直到永远。
 
 目前,有两股科技力量正在相互竞争之中,一股是摧毁生态环境的科技力量,另一股则是拯救生态环境的科技力量。我们正处在人口过多以及过度消费的瓶颈之中。如果这场竞争后者得胜,人类将会进入有史以来最佳的生存状态,而且生物多样性也大致还能保留。
 
 亨利,吾友!谢谢你率先提出这项伦理的第一要义。如今,轮到我们来总结一条更全面的智慧。生物世界正在步向衰亡,自然正在你我繁忙的脚下崩溃。我们人类一向太过热衷于自己的想法,以至于没有预见到我们的行为所造成的长远影响,人类要是再不甩开自己的幻觉,快速谋求解决之道,将来可要损失惨重了。现在,科技一定得帮助我们找寻出路,走出困境。
 
 你曾说过,老习惯适合老人,新行为适合新人。但我认为,就历史的角度看来情况恰恰相反。你是新人,我们是老人。然而,我们现在还能变得更智慧些吗?对于居住在瓦尔登湖畔的你来说,野鸽子的晨间哀歌,青蛙划破黎明水面的呱呱声,就是挽救这片大地的真正理由。对于我们,挽救它则是为了准确掌握事实,探究事实所隐含的意义,以及如何运用事实以达成最佳效果。所以,共有两种事实,你、我以及所有现在的和后来的人,只要接受大自然的主宰,便都会得到。}