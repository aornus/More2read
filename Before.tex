 \section{往期萃取文章}
% \dailyquote{中英}{作者}{封面}
 \poem{The Tyger(老虎)}{William Blake(威廉·布莱克)}{
 	Tyger! Tyger! Burning bright
 	
 	In the forests of the night!
 	
 	What immortal hand or eye
 	
 	Could frame thy fearful symmetry?\\
 	
 	
 	In what distant deeps or skies
 	
 	Burnt the fire of thine eyes?
 	
 	On what wings dare he aspire?
 	
 	What the hand dare seize the fire?\\
 	
 	
 	And what shoulder,  what art
 	
 	Could twist the sinews of thy heart?
 	
 	And when thy heart began to beat,
 	
 	What dread hand,  what dread feet?\\
 	
 	
 	What the hammer? What the chain?
 	
 	In what furnace was thy brain?
 	
 	What the anvil? What dread grasp?
 	
 	Dare its deadly terrors clasp?\\
 	
 	
 	When the stars threw down their spears,
 	
 	And water'd heaven with their tears,
 	
 	Did he smile his work to see?
 	
 	Did he who made the lamb make thee?\\
 	
 	
 	Tyger! Tyger! Burning bright
 	
 	In the forests of the night!
 	
 	What immortal hand or eye
 	
 	Dare frame thy fearful symmetry?\\
 }{老虎!老虎!黑夜的森林中
 
 燃烧着的煌煌的火光,
 
 是怎样的神手或天眼
 
 造出了你这样的威武堂堂?\\
 
 
 
 
 你炯炯的两眼中的火
 
 燃烧在多远的天空或深渊?
 
 他乘着怎样的翅膀搏击?
 
 用怎样的手夺来火焰?\\
 
 
 
 
 又是怎样的膂力,怎样的技巧,
 
 把你的心脏的筋肉捏成?
 
 当你的心脏开始搏动时,
 
 使用怎样猛的手腕和脚胫?\\
 
 
 
 
 是怎样的槌?怎样的链子?
 
 在怎样的熔炉中炼成你的脑筋?
 
 是怎样的铁砧?怎样的铁臂
 
 敢于捉着这可怖的凶神?\\
 
 
 
 
 群星投下了他们的投枪。
 
 用它们的眼泪润湿了穹苍,
 
 他是否微笑着欣赏他的作品?
 
 他创造了你,也创造了羔羊?\\
 
 
 
 老虎!老虎!黑夜的森林中
 
 燃烧着的煌煌的火光,
 
 是怎样的神手或天眼
 
 造出了你这样的威武堂堂?
 
}
 \words{A Haunted House(鬼屋)}{ Virginia Woolf}{ Whatever hour you woke there was a door shutting. From room to room they went, hand in hand, lifting here, opening there, making sure—a ghostly couple.
 	
 	“Here we left it,” she said. And he added, “Oh, but here too!” “It’s upstairs,” she murmured. “And in the garden,” he whispered “Quietly,” they said, “or we shall wake them.”
 	
 	But it wasn’t that you woke us. Oh, no. “They’re looking for it; they’re drawing the curtain,” one might say, and so read on a page or two. “Now they’ve found it,” one would be certain, stopping the pencil on the margin. And then, tired of reading, one might rise and see for oneself, the house all empty, the doors standing open, only the wood pigeons bubbling with content and the hum of the threshing machine sounding from the farm. “What did I come in here for? What did I want to find?” My hands were empty. “Perhaps it’s upstairs then?” The apples were in the loft. And so down again, the garden still as ever, only the book had slipped into the grass.
 	
 	But they had found it in the drawing room. Not that one could ever see them. The window panes reflected apples, reflected roses; all the leaves were green in the glass. If they moved in the drawing room, the apple only turned its yellow side. Yet, the moment after, if the door was opened, spread about the floor, hung upon the walls, pendant from the ceiling—what? My hands were empty. The shadow of a thrush crossed the carpet; from the deepest wells of silence the wood pigeon drew its bubble of sound. “Safe, safe, safe,” the pulse of the house beat softly. “The treasure buried; the room . . . ” the pulse stopped short. Oh, was that the buried treasure?
 	
 	A moment later the light had faded. Out in the garden then? But the trees spun darkness for a wandering beam of sun. So fine, so rare, coolly sunk beneath the surface the beam I sought always burnt behind the glass. Death was the glass; death was between us; coming to the woman first, hundreds of years ago, leaving the house, sealing all the windows; the rooms were darkened. He left it, left her, went North, went East, saw the stars turned in the Southern sky; sought the house, found it dropped beneath the Downs. “Safe, safe, safe,” the pulse of the house beat gladly. “The Treasure yours.”
 	
 	The wind roars up the avenue. Trees stoop and bend this way and that. Moonbeams splash and spill wildly in the rain. But the beam of the lamp falls straight from the window. The candle burns stiff and still. Wandering through the house, opening the windows, whispering not to wake us, the ghostly couple seek their joy.
 	
 	“Here we slept,” she says. And he adds, “Kisses without number.” “Waking in the morning—” “Silver between the trees—” “Upstairs—” “In the garden—” “When summer came—” “In winter snowtime—” The doors go shutting far in the distance, gently knocking like the pulse of a heart.
 	
 	Nearer they come; cease at the doorway. The wind falls, the rain slides silver down the glass. Our eyes darken; we hear no steps beside us; we see no lady spread her ghostly cloak. His hands shield the lantern. “Look,” he breathes. “Sound asleep. Love upon their lips.”
 	
 	Stooping, holding their silver lamp above us, long they look and deeply. Long they pause. The wind drives straightly; the flame stoops slightly. Wild beams of moonlight cross both floor and wall, and, meeting, stain the faces bent; the faces pondering; the faces that search the sleepers and seek their hidden joy.
 	
 	“Safe, safe, safe,” the heart of the house beats proudly. “Long years—” he sighs. “Again you found me.” “Here,” she murmurs, “sleeping; in the garden reading; laughing, rolling apples in the loft. Here we left our treasure—” Stooping, their light lifts the lids upon my eyes. “Safe! safe! safe!” the pulse of the house beats wildly. Waking, I cry “Oh, is this your buried treasure? The light in the heart.”
 	
 	鬼屋
 	弗吉尼亚・伍尔夫
 	
 	无论你何时醒来,总有一扇门关着。他们手牵手,一个房间一个房间地挨个转悠,动动这儿,开开那儿,在确认什么 —— 一对幽灵夫妇。
 	
 	“我们把它留这儿了,”她说。他补充道,“嗯,还有这儿!”“在楼上,”她嘀咕道。 “在花园里,”他小声说。 “轻点儿,”他们说,“不然我们会惊醒他们的。”
 	
 	可你们并没有惊醒我们。呃,没有。 “他们在寻找东西;他们正在拉开窗帘,”有人可能会这么说,于是乎又读上一两页的书。 “现在他们已经找到它了,“有人会这么断定,在书的页边空白处停下铅笔。再者,有人读书倦了,可能会站起身来,对这个空空荡荡的房子亲自察看一番,门是敞开的,只有斑鸠发出的满意的咕咕声和从农场传来的脱粒机的嗡嗡声。“我来这里干什么?我想找到什么?”我双手手空空如也。”也许它在楼上呢?“苹果在阁楼里。然后再次下楼,花园寂静如常,只有书滑落在了草地上。
 	
 	但是他们在客厅找到了它。并不是说有人会看到他们。窗玻璃反射出苹果,反射出玫瑰;在玻璃上,所有的叶子都是绿的。如果他们走进客厅,苹果仅呈现其黄色一面。然而,此时此刻,如果房门被打开,那些叶子的影子就会洒满在地板上,悬挂在墙上,垂吊在天花板上 —— 什么?我双手空空如也。画眉鸟的影子掠过地毯;遥不可测的沉寂的深处,传来斑鸠的咕咕叫声。 “平安,平安,平安,”房子在轻柔脉动。 “埋藏的宝藏;这个房间......”脉动突然停了。哦,那就是埋藏的宝藏?
 	
 	不一会儿,灯光渐渐消失了。那么在外面花园里?可是树木伴随着太阳光束的漫游而喜欢转自己的黑影。我追寻的那束太阳光一直在玻璃后燃烧,如此精美,如此罕见,冷静地沉入地下。死亡是玻璃;死亡就在我们之间,首先来到那个女人身边,数百年前,他离开这座房子,密封了所有的窗户;房间变暗了。他离开了房子,离开了她,奔向北方,奔向东方,看到了南方天空的斗转星移;找寻那座房子,发现它沉降于唐斯丘陵下面。 “平安,平安,平安,”房子高兴地脉动着。 “你的宝藏。”
 	
 	风沿着大街咆哮。树木东倒西歪地扭曲着身躯。大雨,月光缕缕在如注的大雨中飞溅。可是灯的光束却从窗户上直接落下。蜡烛静静地、静静地燃烧着。这对幽灵夫妻寻求着他们的快乐,他们在房子里走来走去,打开窗户,窃窃私语说不要惊醒我们。
 	
 	“我们睡在这里,”她说。他补充道,“亲吻无数。” “早上醒来 ——”“树林间呈银灰色 —— ”“楼上 —— ”“花园里 —— ”“夏天来临之际 —— ”“冬日飞雪时光 —— ”远处的门砰一声关上了,轻柔的叩击声如同心脏的脉动。
 	
 	他们越来越近,停在门口。风力弱了,银色的雨水贴着玻璃向下流淌。我们的眼睛暗淡无光;我们听不到身边的脚步声;我们看不到女士张开的她那幽灵般的斗篷。他双手护着灯笼。 “瞧,”他低声说道。 “他们睡得很香。唇上带着爱意。”
 	
 	他们手握银灯,屈身照耀我们,长久地注视着,深情地注视着。他们久久不肯离去。风儿径直吹来;火苗轻轻摇摆。屡屡月光肆无忌惮地穿越地板和墙壁,会合一处,斑驳陆离地照射着那两张低俯的面孔;沉思的面孔;搜寻酣睡者和寻求潜藏快乐的面孔。
 	
 	“平安,平安,平安,”房子的心脏骄傲地脉动着。 “岁月漫长 —— ”他叹了口气。 “你又找到了我。” “在这里,”她咕哝道,“在睡觉;在花园里,读书;在阁楼里,笑着滚苹果。我们把宝藏留在了这里—— ”他们弯腰时,灯罩脱落,掉到了我的眼睛上。 “平安!平安!平安!”房子疯狂地脉动着。醒来,我哭了“哦,这是你埋藏的宝藏吗?内心深处的灯。”}
 
  \words{  桃花心木( Mahogany)}{   林清玄}{ 乡下老家前面的空地,租给人家种桃花心木的树苗。
  	
  	树苗种下以后,植树人总是隔几天才来浇水。他来的天数并没有规律,有时三天,有时五天,有时十几天才来一次。浇水的量也不一定,有时浇得多,有时浇得少。桃花心木苗有时就莫名其妙地枯萎了,所以,他来的时候总会带几株树苗来补种。
  	
  	我起先以为他太懒,隔那么久才为树浇水。但是,懒人怎么知道有几棵树会枯萎呢?他说:“种树是百年的基业,所以,树木自己要学会在土里找水源。我浇水只是模仿老天下雨,老天下雨是算不准的。如果无法在这种不确定中汲水生长,树苗自然就枯萎了。但是,只要在不确定中找到水源、拼命扎根,长成百年的大树就不成问题了。”
  	
  	种树人语重心长地说:“如果我每天都来浇水,每天定时浇一定量的水,树苗就会养成依赖的心,根就会浮在地表上,无法深入地下,一旦我停止浇水,树苗会枯萎得更多。幸而存活的树苗,遇到狂风暴雨,也会一吹就倒了。”
  	
  	植树者言,使我非常感动,想到不只是树,人也是一样。在不确定中,我们会养成独立自主的心,不会依赖,我们会把很少的养分转化为巨大的能量,努力生长。
  	
  	
  	
  	In front of my old country house, a piece of uncultivated land was rented to others to plant mahogany trees. After the tree saplings were planted, the planter would come over to water them once every few days. His visits were on an irregular basis, an interval of three, sometimes five, or even a dozen of days. The amount of water he used for watering was also varied from time to time, sometimes more, sometimes less. Some mahogany saplings would then be found withered. Hence, whenever he came by, he would bring a few young plants to replace the withering ones.
  	
  	At first, I took him for a lazy man, someone who's too casual to take to heart the time to attend to his seedlings. However, how could someone like that know exactly how many plants were there withering? He said, "It takes a hundred years for trees to establish their firm footing. They need to explore water sources deep down in the soil for themselves. My watering is a simulation for raining, as unpredictable as rainfall. If a sapling can't adjust itself to the environment and find moisture for its own growth, naturally, it goes to wither. However, once it found water supply and settled its root deep in earth, no doubt the sapling would grow into a long-lasting tree."
  	
  	What he said was thought-provoking. He went on to explain, "If I came every day to give them a definite amount of water, the saplings would get used to it and rely on it. Their roots would be easily broken in the storms." remain skin-deep in the surface; there is no need for them to strike deep into the earth. If I happen to stop watering, the saplings would wither badly. Even those lucky enough to survive would topple over when rainstorms came.
  	
  	I was moved by his remarks. I think the same is true for us human beings. When we encounter uncertainties, we would have to rely on ourselves and act on our own. In that way, we would be like the mahogany trees. Turning the little amount of nutrient we have absorbed into powerful energy to help us grow.}
  \words{ The Wind in the Willows(柳林风声)}{Kenneth Grahame(肯尼斯·格雷厄姆)}{ 	
  	The Mole had been working very hard all the morning, spring-cleaning his little home. First with brooms, then with dusters; then on ladders and steps and chairs, with a brush and a pail of whitewash; till he had dust in his throat and eyes, and splashes of whitewash all over his black fur, and an aching back and weary arms. Spring was moving in the air above and in the earth below and around him, penetrating even his dark and lowly little house with its spirit of divine discontent and longing. It was small wonder, then, that he suddenly flung down his brush on the floor, said 'Bother!' and 'O blow!' and also 'Hang spring-cleaning!' and bolted out of the house without even waiting to put on his coat. Something up above was calling him imperiously, and he made for the steep little tunnel which answered in his case to the gravelled carriage-drive owned by animals whose residences are nearer to the sun and air. So he scraped and scratched and scrabbled and scrooged and then he scrooged again and scrabbled and scratched and scraped, working busily with his little paws and muttering to himself, 'Up we go! Up we go!' till at last, pop! his snout came out into the sunlight, and he found himself rolling in the warm grass of a great meadow.
  	
  	整个上午,鼹鼠都在勤奋地干活,为他小小的家屋作春季大扫除,先用扫帚扫,再用掸子掸,然后登上梯子、椅子什么的,拿着刷子,提着灰浆桶,刷墙,直干到灰尘呛了嗓子,迷了眼,全身乌黑的毛皮溅满了白灰浆,腰也酸了,臂也痛了。春天的气息,在他头上的天空里吹拂,在他脚下的泥土里游动,在他四周围飘荡。春天那奇妙的追求、渴望的精神,甚至钻进了他那阴暗低矮的小屋。怪不得他猛地把刷子往地下一扔,嚷道:“烦死人了!”“去它的!”“什么春季大扫除,见它的鬼去吧!”连大衣也没顾上穿,就冲出家门了。上面有种力量在急切地召唤他,于是他向着陡峭的地道奔去。这地道,直通地面上的碎石子大车道,而这车道是属于那些住在通风向阳的居室里的动物的。鼹鼠又掏又挠又爬又挤,又挤又爬又挠又掏,小爪子忙个不停,嘴里还不住地念念叨叨,“咱们上去啰!咱们上去啰!”末末了,噗的一声,他的鼻尖钻出了地面,伸到了阳光里,跟着,身子就在一块大草坪暖暖的软草里打起滚来。
  	
  	is better than whitewashing!' The sunshine struck hot on his fur, soft breezes caressed his heated brow, and after the seclusion of the cellarage he had lived in so long the carol of happy birds fell on his dulled hearing almost like a shout. Jumping off all his four legs at once, in the joy of living and the delight of spring without its cleaning, he pursued his way across the meadow till he reached the hedge on the further side.
  	
  	“太棒了!”他自言自语说,“可比刷墙有意思!”太阳晒在他的毛皮上,暖烘烘的,微风轻抚着他发热的额头,在洞穴里蛰居了那么久,听觉都变得迟钝了,连小鸟儿欢快的鸣唱,听起来都跟大声喊叫一样。生活的欢乐,春天的愉悦,又加上免了大扫除的麻烦,他乐得纵身一跳,腾起四脚向前飞跑,横穿草坪,一直跑到草坪尽头的篱笆前。
  	
  	'Hold up!' said an elderly rabbit at the gap. 'Sixpence for the privilege of passing by the private road!'
  	
  	“站住!”篱笆豁口处,一只老兔子喝道。“通过私人道路,得交六便士!”
  	
  	He was bowled over in an instant by the impatient and contemptuous Mole, who trotted along the side of the hedge chaffing the other rabbits as they peeped hurriedly from their holes to see what the row was about. 'Onion-sauce! Onion-sauce!' he remarked jeeringly, and was gone before they could think of a thoroughly satisfactory reply. Then they all started grumbling at each other. 'How STUPID you are! Why didn't you tell him --' 'Well, why didn't YOU say --' 'You might have reminded him --' and so on, in the usual way; but, of course, it was then much too late, as is always the case.
  	
  	鼹鼠很不耐烦,态度傲慢,根本没把老兔子放在眼里,一时倒把老兔子弄得不知如何是好。鼹鼠顺着篱笆一溜小跑,一边还逗弄着别的兔子,他们一个个从洞口探头窥看,想知道外面到底吵些什么。“蠢货!蠢货!”他嘲笑说,不等他们想出一句解气的话来回敬他,就一溜烟跑得没影儿了。这一来,兔子们七嘴八舌互相埋怨起来。“瞧你多蠢,干吗不对他说……”“哼,那你干吗不说……”“你该警告他……”诸如此类,照例总是这一套。当然啰,照例总是——太晚啦。
  	
  	It all seemed too good to be true. Hither and thither through the meadows he rambled busily, along the hedgerows, across the copses, finding everywhere birds building, flowers budding, leaves thrusting -- everything happy, and progressive, and occupied. And instead of having an uneasy conscience pricking him and whispering 'whitewash!' he somehow could only feel how jolly it was to be the only idle dog among all these busy citizens. After all, the best part of a holiday is perhaps not so much to be resting yourself, as to see all the other fellows busy working.
  	
  	一切都那么美好,好得简直不像是真的。他跑过一片又一片的草坪,沿着矮树篱,穿过灌木丛,匆匆地游逛。处处都看到鸟儿做窝筑巢,花儿含苞待放,叶儿挤挤嚷嚷——万物都显得快乐,忙碌,奋进。他听不到良心在耳边嘀咕:“刷墙!”只觉得,在一大群忙忙碌碌的公民当中,做\textbf{一只唯一的懒狗,是多么惬意}。看来,过休假日最舒心的方面,还不是自己得到休憩,而是看到别人都在忙着干活。
  	
  	He thought his happiness was complete when, as he meandered aimlessly along, suddenly he stood by the edge of a full-fed river. Never in his life had he seen a river before this sleek, sinuous, full-bodied animal, chasing and chuckling, gripping things with a gurgle and leaving them with a laugh, to fling itself on fresh playmates that shook themselves free, and were caught and held again. All was a-shake and a-shiver-glints and gleams and sparkles, rustle and swirl, chatter and bubble. The Mole was bewitched, entranced, fascinated. By the side of the river he trotted as one trots, when very small, by the side of a man who holds one spell-bound by exciting stories; and when tired at last, he sat on the bank, while the river still chattered on to him, a babbling procession of the best stories in the world, sent from the heart of the earth to be told at last to the insatiable sea.
  	
  	他漫无目的地闲逛着,忽然来到一条水流丰盈的大河边,他觉得真是快乐绝顶了。他这辈子还从来没有见过一条河哩。这只光光滑滑、蜿蜿蜒蜒、身躯庞大的动物,不停地追逐,轻轻地欢笑。它每抓住什么,就格格低笑,把它们扔掉时,又哈哈大笑,转过来又扑向新的玩伴。它们挣扎着甩开了它,可到底还是被它逮住,抓牢了。它浑身颤动,晶光闪闪,沸沸扬扬,吐着旋涡,冒着泡沫,喋喋不休地唠叨个没完。这景象,简直把鼹鼠看呆了,他心驰神迷,像着了魔似的。他沿着河边,迈着小碎步跑,像个小娃娃紧跟在大人身边,听他讲惊险故事,听得入了迷似的。他终于跑累了,在岸边坐了下来。可那河还是一个劲儿向他娓娓而谈,它讲的是世间最好听的故事。这些故事发自地心深处,一路讲下去,最终要向那听个没够的大海倾诉。
  	
  	
  	As he sat on the grass and looked across the river, a dark hole in the bank opposite, just above the water's edge, caught his eye, and dreamily he fell to considering what a nice snug dwelling-place it would make for an animal with few wants and fond of a bijou riverside residence, above flood level and remote from noise and dust.
  	
  	As he gazed, something bright and small seemed to twinkle down in the heart of it, vanished, then twinkled once more like a tiny star. But it could hardly be a star in such an unlikely situation; and it was too glittering and small for a glow-worm. Then, as he looked, it winked at him, and so declared itself to be an eye; and a small face began gradually to grow up round it, like a frame round a picture.
  	
  	他坐在草地上,朝着河那边张望时,忽见对岸有个黑黑的洞口,恰好在水面上边。他梦悠悠地想,要是一只动物要求不过高,只想有一处小巧玲珑的河边住宅,涨潮时淹不着,又远离尘嚣,这个住所倒是满舒适的。他正呆呆地凝望,忽觉得,那洞穴的中央有个亮晶晶的小东西一闪,忽隐忽现,像一颗小星星。不过,出现在那样一个地方,不会是星星。要说是萤火虫嘛,又显得太亮,也太小。望着望着,那个亮东西竟冲他眨巴了一下,可见那是一只眼睛。接着,围着那只眼睛,渐渐显出一张小脸,恰像一幅画,嵌在画框里。
  	
  	A brown little face, with whiskers.
  	
  	一张棕色的小脸,腮边有两撇胡鬚。
  	
  	A grave round face, with the same twinkle in its eye that had first attracted his notice.
  	
  	一张神情严肃的圆脸,眼睛里闪着光,就是一开始引起他注意的那种光。
  	
  	Small neat ears and thick silky hair.
  	
  	一对精巧的小耳朵,一头丝一般浓密的毛发。
  	
  	It was the Water Rat!
  	
  	那是河鼠!
  	
  	Then the two animals stood and regarded each other cautiously.
  	
  	随后,两只动物面对面站着,谨慎地互相打量。
  	
  	'Hullo, Mole!' said the Water Rat.
  	
  	“嗨,鼹鼠!”河鼠招呼道。
  	
  	'Hullo, Rat!' said the Mole.
  	
  	“嗨,河鼠!”鼹鼠答道。
  	
  	'Would you like to come over?' enquired the Rat presently.
  	
  	“你愿意过这边来吗?”河鼠问。
  	
  	'Oh, its all very well to TALK,' said the Mole, rather pettishly, he being new to a river and riverside life and its ways.
  	
  	“嗳,说说倒容易,”鼹鼠没好气地说,因为他是初次见识一条河,还不熟悉水上的生活习惯。
  	
  	The Rat said nothing, but stooped and unfastened a rope and hauled on it; then lightly stepped into a little boat which the Mole had not observed. It was painted blue outside and white within, and was just the size for two animals; and the Mole's whole heart went out to it at once, even though he did not yet fully understand its uses.
  	
  	河鼠二话没说,弯腰解开一条绳子,拽拢来,然后轻轻地跨进鼹鼠原先没有注意到的一只小船。那小船外面漆成蓝色,里面漆成白色,鼹鼠的心,一下子飞到了小船上,虽然他还不大明白它的用场。
  	
  	The Rat sculled smartly across and made fast. Then he held up his forepaw as the Mole stepped gingerly down. 'Lean on that!' he said. 'Now then, step lively!' and the Mole to his surprise and rapture found himself actually seated in the stern of a real boat.
  	
  	河鼠干练地把船划到对岸,停稳了。他伸出一只前爪,搀着鼹鼠小心翼翼地走下来。“扶好了!”河鼠说,“现在,轻轻地跨进来!”于是鼹鼠又惊又喜地发现,自己真的坐进了一只真正的小船的尾端。}
  \poem{听瓶记(Listening to a Bottle)}{
  	余光中}{Always had I imagined that all bottles
  	
  	Were empty and mindless
  	
  	Until one day I leaned to a bottle's mouth,
  	
  	Surprised to hear the whole world
  	
  	Whirling and whirling inside
  	
  	Into a rounded perfect song,
  	
  	Just as the clear serenity
  	
  	That settles to the bottom of my mind
  	
  	Was but the world's turbulent din
  	
  	That fell whirling and dashing
  	
  	Shrill against my victim ear.
  	
  	
  }{一直以为全世界所有的瓶
  
  都是空的,无所用心
  
  直到有一天俯向瓶口
  
  惊闻全世界所有的声音
  
  都在瓶底回荡又回荡
  
  听不厌,
  
  隐隐浑圆的妙响
  
  亦如我心底澄澈的宁静
  
  原是举世滔滔
  
  逆耳旋来的
  
  千般噪音}
\words{Suppose Someone Gave You a Pen}{arg2}{Suppose someone gave you a pen---a sealed, solid-colored pen.
	
	You couldn’t see how much ink it had. It might run dry after the first few tentative words or last just long enough to create a masterpiece (or several ) that would last forever and make a difference in the scheme of things. You don’t know before you begin. Under the rules of the game, you really never know .You have to take a chance!
	
	Actually, no rule of the game states you must do anything. Instead of picking up and using the pen, you could leave it on a shelf or in a drawer where it will dry up, unused. But if you do decide to use it, what would you do with it? How would you play the game?
	
	Would you plan and plan before you ever wrote a word? Would your plans be so extensive that you never even got to the writing? Or would you take the pen in hand, plunge right in and just do it, struggling to keep up with the twists and turns of the torrents of words that take you where they take you? Would you write cautiously and carefully, as if the pen might run dry the next moment, or would you pretend or believe (or pretend to believe) that the pen will write forever and proceed accordingly?
	
	And of what would you write: Of love? Hate? Fun? Misery? Life? Death? Nothing? Everything? Would you write to please just yourself? Or others? Or yourself by writing for others? Would you strokes be tremblingly timid or brilliantly bold? Fancy with a flourish or plain? Would you even write? Once you have the pen, no rule says you have to write. Would you sketch? Scribble? Doodle or draw? Would you stay in or on the lines, or see no lines at all, even if they were there? Or are they?
There’s a lot to think about here, isn’t there?

Now, suppose someone gave you a life…


假如有人送你一支笔,一支不可拆卸的单色钢笔。

看不出里面究竟有多少墨水。或许在你试探性地写上几个字后它就会干枯,或许足够用来创造一部(或几部)影响深远的不朽巨著。而这些,在动笔前都是无法得知的。在这个游戏规则下,你真的永远不会预知结果。你只能去碰运气!

事实上,这个游戏里没有规则指定你必须要做点什么。相反的,你甚至可以根本不去动用这支笔,把它扔在书架上或是抽屉里让它的墨水干枯。但是,如果你决定要用它的话,那么你会用它来做什么呢?你将怎么来进行这个游戏呢?

你会在动笔写字之前,老是计划来计划去吗?你会不会因为计划过于宏大而无从动笔呢?或者你只是手里拿着笔,一头扎进去写,不停地写,努力地使自己随着文字汹涌的浪涛而奔流?抑或,你会小心谨慎地写字,好像这支笔在下一个时刻就可能会干枯?还是假装或相信这支笔能够永远写下去而信手写来呢?

你又会用笔写下些什么呢:爱?恨?喜?悲?生?死?虚无?万物?你写作只是为了取悦自己?还是为了取悦他人?抑或是借替人写书之机而愉悦自己?你落笔时会颤抖胆怯,还是敏锐果敢?你的想象是会丰富的还是贫乏的?或者说你真的会动笔吗?你拿到笔以后,并没有哪条规则说你必须写作。也许你要快笔素描,乱写一气?信笔涂鸦?只是激情作画?你会保持写在线内还是线上,还是根本看不到线——即使有线在那里?又或者,真的有线存在吗?

这里面有很多东西值得考虑,不是吗?

现在,假如有人给你一支生命的笔…}
\poem{Dover Beach(多佛海滨)
}{	Matthew Arnold	
(麦修·阿诺德)|孙梁 译}{The sea is calm tonight.
	
	The tide is full, the moon lies fair
	
Upon the straits —on the French coast the light
	
	Gleams and is gone; the cliffs of England stand,
	
	Glimmering and vast, out in the tranquil bay.
	
	Come to the window, sweet is the night air!
	
	Only, from the long line of spray
	
	Where the sea meets the moon-blanched land.\\
	
	
	Listen! you hear the grating roar
	
	Of pebbles which the waves draw back, and fling,
	
	At their return, up the high strand,
	
	Begin, and cease, and then again begin,
	
	With tremulous cadence slow, and bring
	
	The eternal note of sadness in.\\
	
	
	
	Sophocles long ago
	
	Heard it on the Aegean, and it brought
	
	Into his mind the turbid ebb and flow
	
	Of human misery; we
	
	Find also in the sound a thought,
	
	Hearing it by this distant northern sea.\\
	
	The Sea of Faith
	
	Was once, too, at the full, and round earth's shore
	
	Lay like the folds of a bright girdle furled.
	
	But now I only hear
	
	Its melancholy, long, withdrawing roar,
	
	Retreating, to the breath
	
	Of the night wind, down the vast edges drear
	
	And naked shingles of the world.
	
	Ah, love, let us be true
	
	To one another! for the world, which seems
	
	To lie before us like a land of dreams,
	
	So various, so beautiful, so new,
	
	Hath really neither joy, nor love, nor light,
	
	Nor certitude, nor peace, nor help for pain;
	
	And we are here as on a darkling plain
	
	Swept with confused alarms of struggle and flight,
	
	Where ignorant armies clash by night.
}{	

今晚风平浪静,

涨潮时,皓月朗照

海峡;彼岸法兰西,灯影\\


明灭;此地英伦,悬崖耸峙,\\


巍然闪现,俯视海湾如镜。

倚窗兮,夜间清风习习!

惟有浪花蔓延飞溅,

惊涛拍岸,岸上月华如洗。\\


听!浪卷卵石訇砰,

狂澜退,又冲回,翻腾\\


滩头,掀起千堆石;

时起,时伏,消长不息,

幽幽咽咽,袅袅余音

万古愁,不绝如缕。\\



古人索福克勒斯

爱琴海边听涛声,

浊浪滔滔,悠悠哀思:

人间苦难无穷;

千载下,尔我听涛北海滨,

应有同感油然生。\\




信仰之海

昔日汹涌澎湃,

波光似锦带,缭绕寰球。

如今衰朽,

只听得涛音凄恻,

退潮时奄奄一息,

夜风呜咽,荒滩漫无际,

浪去也,席卷平沙顽石。\\



眷恋莫背弃!

眼前红尘迷离,

依稀似幻梦:

新奇、瑰丽、绚烂多姿,

其实无爱,无光,无生趣,

不安,不宁,苦海伶仃;

仿佛荒原阴森,

黑夜里乌合之众交锋,

乱纷纷相争,惶惶然逃遁}